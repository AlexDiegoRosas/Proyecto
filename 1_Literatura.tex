\titlespacing{\chapter}{0pt}{120pt}{7pt}

\chapter{Generalidades De La Empresa}
\label{cap:generalidades}
\authoredby{B}

\section{Razón social}
\label{sec:razonsocial}

Proefex Sociedad Comercial de Responsabilidad Limitada  - Proefex S.R.L.

\section{Misión, Visión, Objetivos, Valores de la empresa.}

\subsection{Misión}

Aumentar el valor de nustros clientes de forma responsable, proponiendo y resolviendo con soluciones 
tecnológicas, innovadoras y creativas.

\subsection{Visión}

Ser una empresa referente en los rubros de tecnología e innovación y marketing en la región sur del Perú.

\subsection{Valores}

\begin{itemize}
	\item Agilidad
	\item Responsabilidad
	\item Sinceridad
	\item Desarrollo e Innovación
    \item Creatividad
	\item Solidaridad
	\item Trabajo en equipo
	\item Puntualidad
\end{itemize}

\section{Servicios, mercado, clientes}
\label{sec:productos} 

\textbf{Servicios}

Proefex es una empresa de tecnología que oefrece a sus clientes los siguientes servicios:

\begin{itemize}
	\item \textbf{Herramientas Martech} \\
		Soluciones tecnológicas especializadas para estrategis de marketing.
	\item \textbf{Web \& E-commerce} \\
        Desarrollo y diseño de sitios web y plataformas de comercio electronico.
	\item \textbf{Automatización y RPA (Automatización de Procesos Robóticos)} \\
		Implementación de sistemas automatizados para mejorar la eficiencia operativa
	\item \textbf{Chatbots}\\
		Creación de sistemas de conversación automatizados para mejorar la atención al
		cliente y la interacción en linea.
	\item \textbf{Desarrollo Full, Low \& no code} \\
		Soluciones de desarrollo adaptadas a diefentes niveles de necesidad.
	\item \textbf{Implementación de IA (Inteligencia Artificial)} \\
        Implementación de sistemas de inteligencia artificial para mejorar la eficiencia operativa.
	\item \textbf{Tour virtual 360} \\
		Creación de recorridos virtuales inmersivos para diversos propósitos, desde inmobiliario hasta turismo.
	\item \textbf{Experiencias VR y AR (Realidad Virtual y Aumentada)}\\
		Desarrollo de experiencias inmersivas mediante tecnologías VR y AR.
	\item \textbf{Drone Show} \\
		Servicios especializados utilizando drones para eventos y presentaciones.
	\item \textbf{Certificaciones}\\
		Programas de certificación en tecnología y habilidades relacionadas.
\end{itemize}

\textbf{Mercado}

Proefex se enfoca en ofrecer sus servicios a empresas líderes y emergentes del sector tecnológico,
		brindando soluciones innovadoras y avanzadas para optimizar sus operaciones y potenciar su
		presencia en un entorno empresarial altamente competitivo.

Enfoque en sectores especificos:

\begin{itemize}
    \item \textbf{Industria Tecnológica}\\ 
	Proefex se dirige a empresas del ámbito tecnológico, ofreciendo herramientas Martech,
	automatización, implementación de IA y desarrollo de software personalizado para satisfacer
	sus necesidades específicas.

    \item \textbf{Negocios Online y Comercio Electronico}\\
	Se enfoca en mejorar las plataformas web, estrategias de eCommerce y soluciones digitales 
	para empresas que buscan destacarse en el mercado online.

    \item \textbf{Eficiencia Operativa}\\
	Proefex proporciona soluciones que mejoran la eficiencia operativa, como herramientas de
	automatización, chatbots y desarrollo sin código, para optimizar procesos internos y la 
	atención al cliente.

    \item \textbf{Experiencias inmersivas}\\
		Dirige sus servicios hacia la creación de experiencias inmersivas, incluyendo tours
		virtuales 360, realidad virtual y aumentada, especialmente para empresas en sectores
		como turismo, bienes raíces y entretenimiento.
\end{itemize}

Apoyo en la Transformación Digital

Proefex ofrece su expertise a empresas en proceso de transformación digital, proporcionando las
herramientas tecnológicas necesarias para adaptarse a las nuevas tendencias y mantenerse a la 
vanguardia en un entorno empresarial en constante evolución.

La especialización de Proefex se centra en ofrecer soluciones tecnológicas de alta calidad
y personalizadas, contribuyendo al crecimiento y éxito de las empresas del sector tecnológico
al satisfacer sus demandas específicas con innovación y excelencia técnica.

\textbf{Clientes}

Los clientes de Proefex son diversos y abarcan los siguientes grupos:

\begin{itemize}
    \item \textbf{Corporaciones de Tecnología} \\
		Empresas líderes en el ámbito tecnológico que buscan soluciones avanzadas para optimizar
		sus procesos y mantenerse a la vanguardia en innovación.
	\item \textbf{Negocios en Proceso de Digitalización} \\
		Compañías que están llevando a cabo procesos de transformación digital y buscan asesoramiento
		y herramientas para adaptarse a las nuevas tendencias tecnológicas.
	\item \textbf{Startups Innovadoras}\\
		Emprendimientos emergentes que requieren soluciones tecnológicas flexibles y ágiles para desarrollar
		y expandir sus operaciones.
	\item \textbf{Industria del Turismo y Bienes Raíces} \\
		Agencias de turismo, empresas inmobiliarias y desarrolladores que buscan soluciones de
		realidad virtual, tours virtuales 360 y herramientas tecnológicas para promover sus servicios.
	\item \textbf{Sector del Entretenimiento} \\
        Compañías que buscan utilizar tecnología avanzada, como drones y experiencias de realidad virtual,
		para eventos especiales y entretenimiento.
	\item \textbf{Instituciones Educativas y de Formación} \\
		Escuelas, universidades o centros de formación interesados en adoptar tecnologías innovadoras para
		mejorar sus métodos de enseñanza y experiencias educativas. 
	\item \textbf{Empresas Online} \\
		Comercios electrónicos y empresas centradas en la web que buscan mejorar sus plataformas, estrategias
		de venta y herramientas de marketing.
\end{itemize}

Estos clientes representan una variedad de industrias y tamaños de empresa que buscan soluciones
tecnológicas personalizadas y avanzadas para mejorar sus operaciones, impulsar su presencia en
línea y mantenerse a la vanguardia en un mercado cada vez más digitalizado.

\section{Estructura de la organzación}

\begin{figure}[!ht]
    \centering
    \begin{tikzpicture}[
        empleado/.style={rectangle, draw, text centered, text width=3.5cm, rounded corners=6pt, drop shadow},
        puesto/.style={text centered},
        seccion/.style={rectangle, draw, text centered, text width=2.5cm, rounded corners=6pt, drop shadow}
    ]
             % CEO & COFOUNDER
        \node[empleado] (ceo) {Claudia Talavera \\ CEO};
        
        % CTO & COFOUNDER
        \node[empleado, below=1.5cm of ceo] (cto) {Duber Campos \\ CTO};

        
        % Secciones
        \node[seccion, below=2cm of cto, xshift=-8cm] (desarrollo) {Desarrollo};
        \node[seccion, below=2cm of cto, xshift=-4cm] (diseno) {Diseño};
        \node[seccion, below=2cm of cto, xshift=0cm] (soporte) {Soporte Digital};
        \node[seccion, below=2cm of cto, xshift=4cm] (community) {Community Manager};
        % \node[seccion, below=2cm of cto, xshift=6cm] (content) {Content Creator}; 
        
        % Desarrollo
        \node[empleado, below=0.5cm of desarrollo] (be) {Back-End Developer};
        \node[empleado, below=0.5cm of be] (fe) {Front-End Developer};
        
        % Diseño
        \node[empleado, below=0.5cm of diseno] (designer1) {Diego Vargas \\ UI Designer};
        \node[empleado, below=0.5cm of designer1] (designer2) {Claudia Luperdiga \\ UI Designer};
        \node[empleado, below=0.5cm of designer2] (designer3) {Cristhi Salinas \\ UI Designer};
        
        % Soporte Digital
        \node[empleado, below=0.5cm of soporte] (sd1) {Renzo Ortega};
        \node[empleado, below=0.5cm of sd1] (sd2) {Bruno Ortega};
        \node[empleado, below=0.5cm of sd2] (sd3) {Jimena Ortega};
        
        % Community Manager
        \node[empleado, below=0.5cm of community] (cm1) {Jorge Awa};
        \node[empleado, below=0.5cm of cm1] (cm2) {Paola Ramos};
        \node[empleado, below=0.5cm of cm2] (cm3) {Sandra Echevarria};
        
        % Conexiones entre los nodos
        \draw[-{Latex[length=3mm]}] (ceo) -- (cto);
        
        \draw[-{Latex[length=3mm]}] (cto) -- (desarrollo);
        \draw[-{Latex[length=3mm]}] (cto) -- (diseno);
        \draw[-{Latex[length=3mm]}] (cto) -- (soporte);
        \draw[-{Latex[length=3mm]}] (cto) -- (community);
        % \draw[-{Latex[length=3mm]}] (cto) -- (content);
        
        \draw[-{Latex[length=3mm]}] (desarrollo) -- (be);
        \draw[-{Latex[length=3mm]}] (be) -- (fe);
        
        \draw[-{Latex[length=3mm]}] (diseno) -- (designer1);
        \draw[-{Latex[length=3mm]}] (designer1) -- (designer2);
        \draw[-{Latex[length=3mm]}] (designer2) -- (designer3);
        
        \draw[-{Latex[length=3mm]}] (soporte) -- (sd1);
        \draw[-{Latex[length=3mm]}] (sd1) -- (sd2);
        \draw[-{Latex[length=3mm]}] (sd2) -- (sd3);
        
        \draw[-{Latex[length=3mm]}] (community) -- (cm1);
        \draw[-{Latex[length=3mm]}] (cm1) -- (cm2);
        \draw[-{Latex[length=3mm]}] (cm2) -- (cm3);
    \end{tikzpicture}
    \caption{Estructura de la organización}
\end{figure}

