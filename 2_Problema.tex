\titlespacing{\chapter}{0pt}{120pt}{7pt}
\chapter{Plan Del Proyecto de Innovación y/o mejora}
\label{cap:problema}
\authoredby{B}

\section{Identificación del problema técnico en la empresa}

Para abordar la identificación del problema, optaremos por emplear la metodología de lluvia
de ideas. A través de esta técnica, recopilaremos ideas y sugerencias provenientes de distintas
perspectivas y experiencias. Posteriormente, analizaremos los resultados obtenidos en las tablas
pertinentes para determinar el problema que requerirá nuestra atención y resolución.

\begin{table}[!ht]
\begin{center}
\begin{tabular}{| c | p{10cm} |}
\hline
\multicolumn{2}{ |c| }{Alex Diego Rosas Quispe} \\ \hline
Item & Problema \\ \hline
1 & Limitada interactividad en la página web actual \\ \hline
2 & Información poco visual y atractiva para los visitantes \\ \hline
3 & Falta de elementos dinámicos para resaltar los servicios ofrecidos \\ \hline
\end{tabular}
\caption{Ideas propuestas por Alex Diego Rosas Quispe}
\label{tab:ideasalex}
\end{center}
\end{table}

\newpage

\begin{table}[!ht]
\begin{center}
\begin{tabular}{| c | p{10cm} |}
\hline
\multicolumn{2}{ |c| }{Brandon Gonzales Tinta} \\ \hline
Item & Problema \\ \hline
1 & Experiencia de usuario desactualizada y poco atractiva \\ \hline
2 & Falta de contenido interactivo para retener la atención del visitante \\ \hline
3 & Ausencia de elementos visuales atractivos para los servicios ofrecidos \\ \hline
\end{tabular}
\caption{Ideas propuestas por Brandon Gonzalez Tinta}
\label{tab:ideasbrandon}
\end{center}
\end{table}

\begin{table}[!ht]
\begin{center}
\begin{tabular}{| p{4cm} | p{5cm} | p{5cm} |}
\hline
\multicolumn{3}{|p{4cm}|}{Tabla de Afinidades} \\ \hline
Ideas Base & \multicolumn{2}{|p {10 cm}|}{Ideas Planteadas} \\ \hline
Limitada interactividad en la página web actual & Experiencia de usuario desactualizada y poco atractiva & Falta de contenido interactivo para retener la atención del visitante \\ \hline
Información poco visual y atractiva para los visitantes & Ausencia de elementos visuales atractivos para los servicios ofrecidos & \\ \hline
Falta de elementos dinámicos para resaltar los servicios ofrecidos & & \\ \hline
\end{tabular}
\caption{Tabla de Afinidades entre Ideas Base e Ideas Planteadas}
\label{tab:afinidades}
\end{center}
\end{table}


\begin{table}[!ht]
\begin{center}
\begin{tabular}{|p{5cm}|c|c|c|}
\hline
\textbf{IDEAS BASE} & \textbf{FRECUENCIA} & \textbf{IMPORTANCIA} & \textbf{FACTIBILIDAD} \\ \hline
Limitada interactividad en la página web actual & 3 & 4 & 3 \\ \hline
Información poco visual y atractiva para los visitantes & 2 & 5 & 4 \\ \hline
Falta de elementos dinámicos para resaltar los servicios ofrecidos & 4 & 3 & 2 \\ \hline
\end{tabular}
\caption{Tabla de Priorización de Ideas Base}
\label{tab:priorizacion}
\end{center}
\end{table}

El análisis de afinidades y priorización de ideas es una parte crucial en la identificación y resolución de problemas. Estas tablas permiten una evaluación estructurada y meticulosa de varias ideas planteadas por diferentes personas, lo que facilita la identificación de patrones comunes, la frecuencia de ocurrencia de ciertos problemas, la importancia percibida de cada problema y su factibilidad para su resolución.

La tabla de afinidades agrupa ideas similares o relacionadas, permitiendo identificar áreas clave que necesitan atención. Por otro lado, la tabla de priorización ayuda a clasificar estas ideas según su frecuencia, importancia y viabilidad, lo que permite enfocarse en las áreas más críticas y determinar qué problemas pueden ser abordados de manera más efectiva y con mayor impacto.

Estos procesos estructurados no solo ayudan a organizar ideas, sino que también brindan un marco para la toma de decisiones más informada y estratégica al priorizar los problemas que requieren atención inmediata y aquellos que pueden abordarse en etapas posteriores.

\section{Objetivos del Proyecto de Innovación y/o Mejora}

\textbf{Objetivo General}

Optimizar la presentación de los servicios a través de la implementación de elementos interactivos,
modelado 2D y animaciones, con el propósito de proporcionar una experiencia más atractiva,
informativa y memorable para los clientes, fortaleciendo así la comunicación efectiva de los
valores y beneficios de los servicios.

\textbf{Objetios Especificos}

\begin{itemize}
\item Crear presentaciones interactivas que involucren a los usuarios. 
\item Utilizar modelado 2D para representar de manera precisa y atractiva los aspectos clave de los servicios ofrecidos. 
\end{itemize}

\section{Antecedentes del Proyecto de Innovación y/o mejora (Investigaciones realizadas)}

\cite{lizarraga2014blended} Blended-learning afectivo y las herramientas interactivas de la Web 3.0: una revisión sistemática de la literatura.

\cite{cordova2017turismo} Turismo, web 2.0 y Comunicación Interactiva en América Latina. Buenas prácticas y tendencias.

\cite{santos2023interactividad} Interactividad, buscabilidad y visibilidad web en periodismo digital galardonado.
   
\section{Justificación del Proyecto de Innovación y/o Mejora}
La presente propuesta de mejora busca fortalecer la presentación de servicios de Proefex a
través de la implementación de herramientas interactivas basadas en modelado 2D y
animaciones. En un mundo cada vez más digitalizado, la capacidad de presentar los
servicios de manera dinámica y atractiva se ha convertido en un factor crucial para
captar la atención y generar un impacto significativo en el mercado. 

Con el fin de mantenerse a la vanguardia en la industria, es fundamental adaptarse a las
demandas y expectativas de los clientes actuales, quienes valoran la interactividad y la
visualización dinámica como medios efectivos para comprender los servicios ofrecidos.
Mediante el uso de modelado 2D y animaciones, Proefex podrá potenciar la presentación de
sus servicios, ofreciendo una experiencia más inmersiva y atractiva para sus potenciales
clientes.

Además, la implementación de estas herramientas no solo mejorará la presentación de
servicios, sino que también permitirá destacar la innovación y el compromiso de Proefex
con la excelencia en la prestación de servicios, consolidando su posición como líder en
el mercado de Arequipa, Perú. Esta iniciativa no solo elevará la percepción de la
empresa, sino que también contribuirá a aumentar la visibilidad y atraer nuevos
clientes, fortaleciendo así su posición competitiva en el sector. 

En resumen, esta propuesta de mejora se fundamenta en la necesidad de adaptación a un
entorno empresarial cada vez más orientado hacia la interactividad y la presentación
dinámica de servicios, permitiendo a Proefex elevar su oferta, mejorar su imagen de
marca y mantener su competitividad en el mercado local y regional.

\section{Marco Teorico y Conceptual}

\subsection{Fundamento teórico del Proyecto de Innovación y Mejora}

 Interactividad y Experiencia del Usuario (UX/UI): La interactividad juega un papel
fundamental en la experiencia del usuario. Teorías de diseño centradas en el usuario,
como la Teoría de la Usabilidad de Nielsen, destacan la importancia de interfaces
interactivas para atraer, retener y comprometer a los usuarios.

 Modelado 2D y Animaciones: En el ámbito del diseño y la presentación, el modelado 2D
 y las animaciones se basan en principios de diseño visual, teorías de percepción y
 psicología del color y la forma. Conceptos como la Ley de la Continuidad Gestáltica
 y la teoría del movimiento en animación respaldan la efectividad de estas técnicas
 para captar la atención y transmitir información de manera efectiva.

 Aprendizaje visual y Memoria: La teoría del aprendizaje visual sostiene que las
 personas tienden a recordar mejor la información cuando se presenta de manera visual
 y dinámica. Esto se relaciona con la teoría de la memoria cognitiva, que sugiere que
 la información visual se procesa y retiene de manera más eficiente que la información
 puramente textual.

 Marketing y Comunicación Visual: Teorías de marketing como el Marketing Sensorial
 respaldan la importancia de estimular los sentidos y crear experiencias memorables
 para influir en las decisiones de compra. La teoría de la Comunicación Visual subraya
 cómo los elementos visuales impactan la percepción y comprensión del público objetivo.

  Tecnología y Tendencias Digitales: La rápida evolución tecnológica y las tendencias
  digitales actuales respaldan la implementación de herramientas interactivas y
  animaciones como una estrategia efectiva para destacar y diferenciar los servicios
  ofrecidos en un mercado cada vez más competitivo.

  La combinación de estos fundamentos teóricos respalda la validez y eficacia del uso
  de modelado 2D y animaciones para mejorar la presentación de servicios, ofreciendo
  una experiencia más atractiva, memorable y efectiva para los clientes potenciales de
  Proefex.

\subsection{Conceptos y términos utilizados}

\begin{itemize}
    \item Interactividad: Capacidad de los usuarios para interactuar con una interfaz, sistema o contenido digital, permitiendo acciones y respuestas bidireccionales.
    \item Modelado 2D: Proceso de crear representaciones bidimensionales de objetos, entornos o diseños utilizando software especializado.
    \item Animaciones: Secuencias de imágenes en movimiento creadas mediante la manipulación y reproducción de una serie de cuadros estáticos.
    \item Experiencia del Usuario (UX): Enfoque en el diseño de productos o servicios centrado en la satisfacción y facilidad de uso percibida por los usuarios al interactuar con ellos. \cite{ferrer2023aplicabilidad}
    \item Diseño Visual: Aplicación de principios y técnicas visuales para comunicar ideas o conceptos de manera efectiva y atractiva.
    \item Usabilidad: Grado en el que un producto o sistema puede ser utilizado por usuarios específicos para alcanzar objetivos específicos con eficacia, eficiencia y satisfacción en un contexto específico de uso.
    \item Teoría de la Usabilidad (Nielsen): Marco teórico desarrollado por Jakob Nielsen que establece principios y pautas para mejorar la usabilidad de los productos digitales. \cite{preciado2023analisis} 
    \item Ley de la Continuidad Gestáltica: Principio de percepción visual que postula que los elementos visuales tienden a ser percibidos de manera continua cuando se alinean o continúan en una dirección específica. \cite{ciafardo2020breviario}
    \item Psicología del Color: Estudio de cómo los colores afectan la percepción y el comportamiento humano. \cite{garcia2023psicologia}
    \item Teoría del Aprendizaje Visual: Concepto que sugiere que las personas retienen mejor la información cuando se presenta visualmente en lugar de solo texto. \cite{penaherrera2023paratextualidad}
    \item Marketing Sensorial: Estrategia de marketing que busca estimular los sentidos del consumidor para influir en sus emociones y decisiones de compra.
    \item Comunicación Visual: Uso de elementos visuales para transmitir información, ideas o mensajes de manera efectiva.
    \item Tecnologías Emergentes: Nuevas tecnologías que están surgiendo o ganando prominencia en un campo específico.
\end{itemize}

