\titlespacing{\chapter}{0pt}{120pt}{7pt}
\chapter{PLANTEAMIENTO DEL PROBLEMA}
\label{cap:problema}
\authoredby{B}

Esta sección contiene la definición del problema, la justificación, los objetivos y las hipótesis de investigación.
El planteamiento del problema y la postura epistémica deben ser claros, concisos, confiables y concordantes. En este sentido, expresa la relación entre el problema de investigación, la pregunta de investigación y la intención para investigar.


\section{Identificación del problema}

\section{Enunciados del problema}

\section{Justificación}
    
    En la justificación, se fundamentan la contribución a los vacíos de conocimiento y a la resolución del problema de investigación. Los objetivos de la investigación comprenden la intención del estudio y permiten describir el tipo de proceso que ha seguido la investigación. 
    
\section{Objetivos}
    \subsection{Objetivo general}
    \begin{adjustwidth}{2em}{0pt}
    Objetivo general
    \end{adjustwidth}
    
    
    \subsection{Objetivos específicos}
    \begin{adjustwidth}{2em}{0pt}
    \begin{itemize}
        \item Objetivo específico 1
        \item Objetivo específico 2
    \end{itemize}
    \end{adjustwidth}
    
    
\section{Hipótesis}

    Las hipótesis dan cuenta de las respuestas probables al problema objeto de investigación.