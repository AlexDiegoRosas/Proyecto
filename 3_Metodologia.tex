\titlespacing{\chapter}{0pt}{120pt}{7pt}
\chapter{MATERIALES Y MÉTODOS}
\label{cap:metodologia}
\authoredby{B}

\section{Lugar de estudio}

Se debe identificar el lugar donde se realizó la investigación y la georreferencia. Así como, las características ambientales, socioeconómicas y culturales. Sustentar la importancia de la zona de estudio y de sus actores.

\section{Población}

La población es un grupo de sujetos u objetos con unas características definitorias diversas. El investigador de acuerdo a los objetivos define los criterios de inclusión y exclusión.

\section{Muestra}

La muestra es representativa de la población. Por ello, se describe la técnica de muestreo adecuada.

\section{Método de Investigación}

El investigador explica el método de investigación según los objetivos y las variables de la investigación: descriptiva, explicativa o experimental.


\section{Descripción detallada de métodos por objetivos específicos}

Considerar en la presentación de la metodología: a) Descripción de variables analizadas en los objetivos específicos, b) Descripción detallada del uso de materiales, equipos, instrumentos, insumos, entre otros y c) Aplicación de prueba estadística inferencial.