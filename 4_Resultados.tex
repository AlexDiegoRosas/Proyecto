\titlespacing{\chapter}{0pt}{120pt}{7pt}
\chapter{Propuesta Técnica de la Mejora }
\label{cap:resultados}
\authoredby{B}

\section{Plan de acción de la mejora propuesta}

En esta sección, presentamos el Plan de Acción detallado para la implementación de mejoras en la presentación de servicios en la página web de Proefex. Este plan se centra en la integración de modelos 2D, animaciones interactivas y elementos de juego para enriquecer la experiencia del usuario y optimizar la presentación de servicios.

El siguiente cuadro resume las acciones propuestas, actividades correspondientes, así como los responsables directos y los encargados del seguimiento. Cada acción está diseñada para abordar áreas específicas y estratégicas, asegurando un enfoque efectivo para lograr una presentación más atractiva, interactiva y envolvente de los servicios ofrecidos por Proefex en su plataforma en línea.

Este plan se ha estructurado con el objetivo de ofrecer una guía clara y detallada para el equipo involucrado en la implementación de estas mejoras. Cada acción y actividad ha sido cuidadosamente asignada a los miembros del equipo con el fin de garantizar un seguimiento efectivo y un progreso continuo hacia la consecución de nuestros objetivos de mejora.

\newpage
Plan 

\begin{table}[ht]
\centering
\begin{tabular}{|m{3cm}|m{4cm}|m{3cm}|m{4cm}|}
\hline
\textbf{Acciones de Mejora} & \textbf{Actividades} & \textbf{Responsable} & \textbf{Responsable de Seguimiento} \\
\hline
Creación de modelos 2D & Desarrollar conceptos y diseños para modelos 2D interactivos. & Equipo de Diseño & Gerente de Proyecto \\
\hline
Implementación de animaciones & Integrar animaciones interactivas en la página web. & Equipo de Desarrollo Web & Coordinador de Proyecto \\
\hline
Desarrollo de elementos de juego & Diseñar y programar elementos interactivos tipo juego. & Equipo de Desarrollo & Líder de Desarrollo \\
\hline
Pruebas de funcionalidad & Realizar pruebas exhaustivas de los nuevos elementos en entornos de usuario reales. & Equipo de Control de Calidad & Coordinador de Pruebas \\
\hline
Optimización y ajustes & Realizar mejoras y ajustes según comentarios y métricas de usuario. & Equipo de Desarrollo & Gerente de Experiencia de Usuario \\
\hline
Seguimiento y análisis & Monitorear el desempeño y la recepción de los nuevos elementos interactivos. & Equipo de Analítica Web & Analista de Datos \\
\hline
\end{tabular}
\caption{Plan de Acción de la Mejora Propuesta}
\end{table}

\section{Consideraciones técnicas, operativas y ambientales para la implementación de la mejora}

En esta sección, detallamos las consideraciones técnicas, operativas y ambientales fundamentales para la exitosa implementación de las mejoras propuestas en la presentación de servicios en la página web de Proefex. Estas consideraciones juegan un papel crítico en la planificación y ejecución de las actualizaciones, garantizando una transición sin contratiempos hacia una plataforma más interactiva y enriquecida.

Cada área de consideración tiene como objetivo asegurar una implementación efectiva y eficiente, así como minimizar los impactos negativos en la experiencia del usuario y la funcionalidad de la plataforma.

\newpage
\begin{table}[ht]
\centering
\begin{tabular}{|m{3cm}|m{9cm}|}
\hline
\textbf{Tipo de Consideración} & \textbf{Consideraciones} \\
\hline
Técnicas & \begin{itemize}
             \item Requisitos de hardware y software para la integración de modelos 2D y animaciones.
             \item Capacidad de la plataforma web existente para soportar la carga adicional de contenido interactivo.
             \item Tiempo estimado para el desarrollo e implementación de las mejoras técnicas.
          \end{itemize} \\
\hline
Operativas & \begin{itemize}
                \item Capacitación del personal en el manejo y mantenimiento de los nuevos elementos interactivos.
                \item Planificación de pruebas exhaustivas para garantizar la estabilidad y funcionalidad de las mejoras.
                \item Procedimientos de respaldo y recuperación en caso de inconvenientes técnicos.
             \end{itemize} \\
\hline
Ambientales & \begin{itemize}
                 \item Impacto en la experiencia del usuario: asegurar que las mejoras no afecten negativamente la navegación y la usabilidad.
                 \item Consideraciones de rendimiento para garantizar una carga rápida y eficiente de los nuevos elementos.
                 \item Aspectos de diseño y accesibilidad para una experiencia inclusiva.
              \end{itemize} \\
\hline
\end{tabular}
\caption{Consideraciones para la Implementación de la Mejora}
\end{table}

\newpage

\section{Recursos técnicos para implemmentar la mejora propuesta}

En la planificación de la implementación de las mejoras propuestas en la página web de Proefex, es esencial considerar los recursos técnicos necesarios para llevar a cabo con éxito estas actualizaciones. Estos recursos abarcan una variedad de aspectos, desde el talento humano hasta el software, hardware y el tiempo estimado para la ejecución de los cambios.

La siguiente tabla resume los recursos técnicos identificados como fundamentales para la integración de modelos 2D, animaciones interactivas y elementos de juego en la plataforma en línea de Proefex. Cada recurso se describe brevemente, junto con la cantidad estimada o la descripción de su naturaleza.

\begin{table}[ht]
\centering
\begin{tabular}{|m{3cm}|m{6cm}|m{3cm}|}
\hline
\textbf{Recursos} & \textbf{Descripción} & \textbf{Cantidad} \\
\hline
Humanos & Equipo de diseño y desarrollo para crear modelos 2D, animaciones y elementos interactivos. & 8 personas \\
\hline
Software & Plataformas de diseño gráfico (Adobe Illustrator, Sketch), herramientas de animación (Adobe After Effects, Blender) y software de desarrollo web (HTML5, CSS, JavaScript). & - \\
\hline
Hardware & Computadoras de alto rendimiento para el diseño y desarrollo de los elementos interactivos. & 5 equipos \\
\hline
Tiempo & Estimación de 4 meses para el diseño, desarrollo, pruebas y lanzamiento de las mejoras. & - \\
\hline
Presupuesto & Fondos destinados a adquirir software, hardware y posiblemente capacitaciones. & \$25,000 \\
\hline
\end{tabular}
\caption{Recursos Técnicos para la Implementación de la Mejora}
\end{table}

\newpage
\section{Diagrama del proceso, mapa del flujo de valor y/o diagrama de operación de la situación mejorada}

En la representación gráfica del flujo de navegación dentro de la página web, se observa una estructura renovada que enfatiza la sección de servicios. En esta nueva propuesta, la sección de "Servicios" se ha enriquecido con modelos 2D y animaciones, añadiendo un aspecto interactivo que potencia la experiencia del usuario al explorar los servicios ofrecidos por la plataforma. Esta representación visual ilustra cómo la incorporación de modelos 2D y animaciones se integra en el flujo general del sitio web, brindando una experiencia más dinámica e interactiva para los usuarios.

\begin{figure}[!ht]
\centering
\tikzstyle{circleblock} = [circle, draw, text width=1.5cm, text centered, minimum height=1.5cm]
\tikzstyle{squareblock} = [rectangle, draw, text width=1.5cm, text centered, minimum height=1.5cm]
\tikzstyle{line} = [-{Stealth[length=5mm,width=3mm]}, thick]

\begin{tikzpicture}[node distance=3cm, auto]
  \node [circle, draw, text width=2cm, align=center] (inicio) {Inicio};
  \node [circle, draw, text width=2cm, align=center, right of=inicio] (servicios) {Servicios\\Modelos 2D\\y Animaciones};
  \node [circle, draw, text width=2cm, align=center, below of=servicios] (eventos) {Eventos};
  \node [circle, draw, text width=2cm, align=center, below of=inicio] (nosotros) {Nosotros};
  \node [circle, draw, text width=2cm, align=center, below of=nosotros] (contacto) {Contacto};
  \node [circle, draw, text width=2cm, align=center, below of=contacto] (blog) {Blog};
  \node [rectangle, draw, text width=2cm, align=center, right of=servicios, xshift=1.5cm] (recursos) {Recursos};

  \draw [line] (inicio) -- (servicios);
  \draw [line] (servicios) -- (eventos);
  \draw [line] (eventos) -- (nosotros);
  \draw [line] (nosotros) -- (contacto);
  \draw [line] (contacto) -- (blog);
  \draw [line] (blog) -- (recursos);
  \draw [line] (recursos) -- (servicios);
\end{tikzpicture}
\end{figure}

\newpage
\section{Cronograma de ejecución de la mejora}

En el siguiente diagrama de Gantt se detallan las acciones planificadas para la mejora propuesta en el proyecto. Cada acción se ha asignado a un intervalo de semanas correspondiente a su duración estimada. \cite{giraldo2023mejoramiento}

\begin{figure}[!ht]
    \centering
    \begin{ganttchart}[
        y unit title=0.4cm,
        y unit chart=0.5cm,
        vgrid,
        hgrid, 
        title label anchor/.style={below=-1.6ex},
        title left shift=.05,
        title right shift=-.05,
        title height=1,
        progress label text={},
        bar height=0.7,
        group right shift=0,
        group top shift=.6,
        group height=.3,
        bar/.append style={fill=blue!50}
        ]{1}{18} % Modificado el rango para terminar en la semana 18
        %labels
        \gantttitle{Semanas}{18} \\
        \gantttitle{1}{1} 
        \gantttitle{2}{1} 
        \gantttitle{3}{1} 
        \gantttitle{4}{1} 
        \gantttitle{5}{1} 
        \gantttitle{6}{1} 
        \gantttitle{7}{1} 
        \gantttitle{8}{1} 
        \gantttitle{9}{1} 
        \gantttitle{10}{1} 
        \gantttitle{11}{1} 
        \gantttitle{12}{1} 
        \gantttitle{13}{1} 
        \gantttitle{14}{1} 
        \gantttitle{15}{1} 
        \gantttitle{16}{1} 
        \gantttitle{17}{1} 
        \gantttitle{18}{1} \\
        %tasks
        \ganttbar{Creación de modelos 2D}{1}{3} \\
        \ganttbar{Implementación de animaciones}{4}{6} \\
        \ganttbar{Desarrollo de elementos de juego}{7}{9} \\
        \ganttbar{Pruebas de funcionalidad}{10}{12} \\
        \ganttbar{Optimización y ajustes}{13}{15} \\
        \ganttbar{Seguimiento y análisis}{16}{18}
    \end{ganttchart}
    \caption{Gantt Chart}
\end{figure}

\section{Aspectos limitantes para la implementación de la mejora}

En el proceso de mejora de la página web de Proefex para incluir modelos 2D, animaciones interactivas y elementos de juego, se identificaron ciertos aspectos que podrían presentar desafíos en la ejecución exitosa de esta mejora. La siguiente tabla detalla estos aspectos, proporcionando una visión de los posibles obstáculos y sus indicadores asociados que podrían afectar la implementación y el rendimiento de esta mejora específica.


\begin{table}[!ht]
\centering
\begin{tabular}{|c|p{4cm}|p{7cm}|}
\hline
\textbf{Item} & \textbf{Aspectos Observados} & \textbf{Indicador} \\
\hline
01 & Limitaciones Tecnológicas & Incapacidad para implementar ciertas funcionalidades debido a la tecnología actual. \\ \hline
02 & Recursos Humanos Insuficientes & Escasez de personal capacitado para manejar nuevas tecnologías. \\ \hline
03 & Restricciones de Presupuesto & Recursos financieros limitados para inversiones en herramientas o personal adicional. \\ \hline
04 & Tiempo de Desarrollo Prolongado & Retrasos en la implementación debido a la complejidad de las nuevas funciones. \\ \hline
05 & Experiencia de Usuario Insatisfactoria & Problemas de rendimiento o dificultades en la interacción con modelos y animaciones. \\  \hline
\end{tabular}
\caption{Aspectos Limitantes para la Implementación de la Mejora}
\end{table}
