\titlespacing{\chapter}{0pt}{120pt}{7pt}
\chapter{Evaluación Técnica y Económica de la Mejora}
\label{cap:evaluación}
\authoredby{B}

\section{Benefico técnico y/o econónico esperado de la mejora}

El beneficio técnico y económico esperado de la mejora en la presentación de servicios en la página web de Proefex se traduce en múltiples aspectos:

\textbf{Beneficio Técnico}

\begin{itemize}
\item \textbf{Mejora de Experiencia del Usuario:} La implementación de modelos 2D,
	animaciones interactivas y elementos tipo juego en la presentación de servicios
	en el sitio web mejorará la experiencia del usuario, haciéndola más atractiva y
	fácil de comprender.
\item \textbf{Aumento de Interacción:} La inclusión de elementos interactivos puede fomentar
la interacción del usuario con el contenido, lo que posiblemente aumente el tiempo
de permanencia en el sitio.
\item \textbf{Actualización Tecnológica:} La adopción de nuevas tecnologías en la presentación
de servicios coloca a la empresa en una posición más actualizada y competitiva en el
mercado.
\end{itemize}

\textbf{Beneficio Económico}

\begin{itemize}
\item \textbf{Atracción de Clientes Potenciales:} Una presentación más atractiva y dinámica
puede atraer a más clientes potenciales, lo que puede aumentar las conversiones y,
	  en última instancia, los ingresos.
\item \textbf{Retención de Clientes:} Una experiencia mejorada puede contribuir a la lealtad del
cliente y a una mayor retención, reduciendo posiblemente la tasa de rebote y aumentando
el retorno a largo plazo.
\item \textbf{Diferenciación Competitiva:} Al ofrecer una experiencia única y visualmente
impactante, la empresa puede destacar entre sus competidores, generando así un valor
adicional.
\end{itemize}

Estos beneficios, tanto en el ámbito técnico como económico, tienen el potencial de impactar positivamente en la percepción de la marca, el compromiso del usuario y, en última instancia, en el rendimiento financiero de la empresa.

\section{Relación Beneficio / Costo}

\cite{perez2023metodos} La relación Beneficio / Costo (B/C) se calcula mediante la fórmula:

\[
\text{B/C} = \frac{\text{Beneficio Total}}{\text{Costo Total}}
\]

\[
\text{B/C} = \frac{{19500 + 25900 + 1200}}{{19500 + 25900}} = \frac{46600}{45400} \approx 1.026
\]

Por tanto, la relación beneficio/costo calculada es aproximadamente 1.026, lo que indica que el proyecto tiene un rendimiento favorable en comparación con los costos incurridos. Esto sugiere una viabilidad positiva para la implementación de la mejora propuesta en términos de beneficios en relación con los costos asociados.

