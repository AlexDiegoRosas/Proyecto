\titlespacing{\chapter}{0pt}{120pt}{7pt}
\chapter{Conclusiones}
\label{cap:concluciones}
\authoredby{B}

\section{Conclusiones respecto a los objetivos del Proyecto de Mejora}

El proyecto de mejora enfocado en la actualización de la presentación de servicios en la página web de Proefex ha logrado significativos avances. Se han implementado modelos 2D, animaciones interactivas y una experiencia de usuario similar a un juego, lo que ha permitido una visualización más dinámica y atractiva de los servicios ofrecidos.

A lo largo del desarrollo del proyecto, se ha observado un incremento en la interacción de los usuarios con la página, lo que sugiere un mayor interés y compromiso por parte de la audiencia. La adición de elementos interactivos ha generado una mayor retención de los visitantes, proporcionando una experiencia más inmersiva y efectiva.

Si bien los costos asociados con la implementación fueron considerables, el proyecto ha logrado generar un valor agregado notable para la empresa. La combinación de elementos visuales atractivos y una navegación más interactiva ha mejorado significativamente la presentación de los servicios, lo que probablemente resultará en un aumento de clientes potenciales y conversiones.

Es importante continuar monitoreando y analizando la recepción y el impacto de estas mejoras a largo plazo para evaluar su efectividad y justificar completamente la inversión realizada. Sin embargo, se espera que estas actualizaciones proporcionen una ventaja competitiva significativa para Proefex en su mercado objetivo.


