\titlespacing{\chapter}{0pt}{120pt}{7pt}
\chapter{Recomendaciones}
\label{cap:recomendar}
\authoredby{B}

\section{Recomendaciones para la empresa respecto del Proyecto de Mejora}

Basado en los resultados obtenidos y la evolución del proyecto de mejora en la presentación de servicios en la página web, se sugieren las siguientes recomendaciones para Proefex:

1. Continuar el monitoreo: Mantener un seguimiento regular de las métricas clave de rendimiento web para evaluar la efectividad continua de las mejoras implementadas. Esto permitirá realizar ajustes y optimizaciones adicionales según sea necesario.

2. Recopilación de comentarios: Implementar sistemas de retroalimentación y encuestas para obtener comentarios directos de los usuarios sobre la nueva experiencia web. Esto proporcionará información valiosa para identificar áreas de mejora y satisfacer mejor las necesidades de los usuarios.

3. Optimización móvil: Asegurarse de que la experiencia del usuario en dispositivos móviles esté completamente optimizada, ya que un número significativo de usuarios acceden a la web a través de sus dispositivos móviles.

4. Capacitación del personal: Brindar capacitación continua al personal encargado de mantener y actualizar la página web. Esto garantizará que se utilicen todas las funciones y características implementadas de manera efectiva.

\newpage
5. Evolución constante: Estar al tanto de las últimas tendencias y tecnologías web para mantener la página actualizada y competitiva. La evolución constante es crucial para mantener el atractivo y la relevancia de la página web en un entorno digital cambiante.

Estas recomendaciones se centran en la optimización continua, el compromiso del usuario y la adaptación a las tendencias emergentes, lo que permitirá a Proefex mantener una presencia en línea sólida y atractiva.


