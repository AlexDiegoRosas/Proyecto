\begin{resumen}

\textbf{Objetivos del Proyecto:}

El proyecto se centra en mejorar la presentación de servicios en la página web de Proefex mediante la integración de modelos 2D, animaciones interactivas y elementos de juego. Los objetivos específicos incluyen enriquecer la experiencia del usuario, aumentar la interactividad, y mejorar la visualización de los servicios ofrecidos por la empresa.

\textbf{Antecedentes:}

El entorno digital actual exige presentaciones más dinámicas y atractivas para captar y retener la atención del usuario. En respuesta a esta demanda, se identificó la necesidad de renovar la presentación de servicios en la página web de Proefex para mantenerse actualizados y competitivos en el mercado.

\textbf{Análisis de la Mejora:}

La implementación de modelos 2D, animaciones interactivas y elementos de juego proporcionará una experiencia más atractiva y envolvente para los visitantes del sitio web. Esto ayudará a destacar los servicios ofrecidos, facilitar la comprensión de la información, y aumentar el tiempo de permanencia en la página, lo que potencialmente impulsará las conversiones y la retención de usuarios.

\textbf{Plan Propuesto:}

El proyecto se dividirá en fases, que incluyen el diseño conceptual, el desarrollo técnico de modelos y animaciones, la integración de elementos de juego, y la fase de pruebas para garantizar la funcionalidad y la experiencia del usuario. Se asignará un equipo multidisciplinario para llevar a cabo estas tareas de manera eficiente y coordinada.

\textbf{Resultados Económicos:}

Si bien la inversión inicial en el proyecto puede ser considerable, se espera que los beneficios a largo plazo superen los costos. La mejora en la presentación de servicios puede atraer a un mayor número de clientes potenciales, aumentar las conversiones y la retención de usuarios, lo que en última instancia puede traducirse en un crecimiento económico sostenido para Proefex.

Este proyecto busca no solo modernizar la presentación de servicios en el sitio web de Proefex, sino también mejorar la experiencia del usuario y potenciar el rendimiento económico de la empresa en un entorno digital competitivo.
\end{resumen}

