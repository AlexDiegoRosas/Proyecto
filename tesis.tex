%%%%%%%%%%%%%%%%%%%%%%%%%%%%%%%%%%%%%%%%%%%%%%%%%%%%%%%%%%%%%
% Plantilla para el Formato EPG cuantitativo de la 
% Escuela de POSGRADO DE LA UNIVERSIDAD NACIONAL DEL
% ALTIPLANO - PUNO. 
%
% La presente plantilla sigue las indicaciones del 
% siguiente enlace: https://drive.google.com/file/d/1yYM7YBnZbC7JnQ6O58tYuQ2SeFgz1ebW/view
% se recomienda revisar el formato base y la Resolución Directoral  
% Nº 0450-2020-DG-EPG-UNA-PUNO, para tener mayores detalles del 
% formato. 
%
% La plantilla utiliza el formato APA 7. Y esta dividido según los 
% capítulos solicitados por la Escuela de Posgrado-UNA-PUNO. Para editar
% la dedicatoria, agradecimientos y demás secciones y capítulos se debe 
% agregar el contenido en la pestaña correspondiente.
% Para añadir figuras se sugiere agregarlos en la carpeta gráficos.
% Para añadir referencias se realiza en mybib.bib
%
%Author Danitza Bermejo (danitza.bermejo@gmail.com)
%
%%%%%%%%%%%%%%%%%%%%%%%%%%%%%%%%%%%%%%%%%%%%%%%%%%%%%%%%%%%%%
% tamaño A4 , tamaño de fuente:12
\documentclass[a4paper,openany,12pt,oneside,hidelinks]{book}

% Tipo de letra: Times New Román en todo el texto del informe de tesis.
\usepackage[spanish,es-lcroman,es-tabla]{babel}
\usepackage{mathptmx}
\usepackage{graphicx,enumitem}
\usepackage[utf8]{inputenc}
\usepackage{ae}

%Márgenes: derecho, superior e inferior 2,5 cm. e izquierdo 3,5 cm
\usepackage[left=3.5cm,right=2.5cm,top=2.5cm,bottom=2.5cm]{geometry}

%Otras librerías utilizadas
\usepackage{acronym}
\usepackage{xspace}
\usepackage{hlundef}
\usepackage{tesis}
\usepackage{setspace}
\usepackage{lineno}
\usepackage{hyperref}
\usepackage{xpatch}
\usepackage{etoc}
\usepackage[utf8]{inputenc}
\usepackage[T1]{fontenc}

\newcommand*\SpecialDocumentsTocHeading
{\vspace{4ex}\parbox[t]{\textwidth}{.}\par
\global\let\SpecialDocumentsTocHeading\empty }

%Modificar Anexos
\usepackage{quotchap}
\usepackage[titles]{tocloft}
\renewcommand{\cftdot}{}
\renewcommand\cftchappagefont{\normalfont} %normal fond page
\usepackage{appendix}
\newlistof{appendixchapter}{apx}{ÍNDICE DE ANEXOS}

%%%%Para APA7th edition
 \usepackage{csquotes}
 \usepackage[style=apa,natbib=true,sortcites=true,sorting=nyt,backend=biber]{biblatex}
 \DefineBibliographyStrings{spanish}{andothers={\textit{et al}\adddot}}
\DeclareLanguageMapping{spanish}{spanish-apa}
\addbibresource{tesis.bib}

%%%% interlineado contenido
\usepackage[titles]{tocloft}
\setlength{\cftbeforechapskip}{3pt}

%%%tablas
\usepackage{color,soul}
\usepackage{etoolbox}
\usepackage{amsmath}
\usepackage{mathtools}
\usepackage{float} %H de las tablas
\usepackage{longtable} %tablas largas
\usepackage[spanish,onelanguage,ruled,vlined]{algorithm2e}

\usepackage[labelsep=period]{caption}
\usepackage{subcaption}
\usepackage{multicol}
\usepackage{multirow}
\usepackage{verbatim}
\usepackage[table,xcdraw]{xcolor} % para resaltar en las tablas
\usepackage{pdfpages} %añadir pdf
\usepackage{wrapfig} %para poner imagenes con texto
\usepackage{lscape}
\usepackage[explicit, pagestyles]{titlesec}
\usepackage{tabularx}
\usepackage{booktabs}
\newcolumntype{C}{>{\centering\arraybackslash}X} % centered version of "X" type
\setlength{\extrarowheight}{1pt}
\usepackage{makecell}
\newcommand{\ra}[1]{\renewcommand{\arraystretch}{#1}}
\usepackage{pifont} % DING
\usepackage{xcolor} % color 
\hypersetup{citecolor=blue}
\hypersetup{urlcolor=red}
\renewcommand\linenumberfont{\normalfont\small}

% interlineado del documento (1.5)
\renewcommand{\baselinestretch}{1.5}
\renewcommand{\arraystretch}{1.0}% interlineado tablas
\usepackage{multirow} %multi
\usepackage{verbatim} % comentarios
\usepackage{tikz}
\usetikzlibrary{positioning, shapes, shadows, arrows.meta}
\usepackage{pgfplots}
\usetikzlibrary{calc}
\usepackage{pgfgantt}
% \pgfplotsset{compat=1.17}

% BORRAR hipervínculo de referencias e hipervínculos de enlaces
\newcommand{\defineauthorcolor}[2]{%
  \colorlet{author#1}{#2}% Create an author colour
  \expandafter\def\csname authoredby#1\endcsname{% Create author colour settings
    \renewcommand{\cftchapfont}{\normalfont\color{author#1}}% Chapter colour
    \renewcommand{\cftchappagefont}{\normalfont\color{author#1}}% Chapter colour
    \renewcommand{\cftsecfont}{\normalfont\color{author#1}}% Section colour
    \renewcommand{\cftsubsecfont}{\normalfont\color{author#1}}}% Subsection colour
}
\makeatletter
\newcommand{\authoredby}[1]{\addtocontents{toc}{\protect\@nameuse{authoredby#1}}}%
\makeatother
\defineauthorcolor{A}{white}% Author A will be coloured red
\defineauthorcolor{B}{black}% Author B will be coloured blue
\def\UrlBreaks{\do\/\do-}

\newenvironment{figrow}%
{%
	\centering\addtocounter{figure}{1}% if caption at bottom
	\begin{enumerate}[%
		itemsep=2pt,parsep=0em,
		label={(\alph*)},
		ref={(\alph*)}]
}%
{\end{enumerate}\addtocounter{figure}{-1}}
\newcommand\img[1]
{\raisebox
	{\dimexpr-0.5\height+0.5ex}
	{\includegraphics[width=0.225\textwidth]{#1}}
}

%Ubicación texto en Figura
\newcounter{row}
\renewcommand\therow{\thefigure\alph{row}}
\newenvironment{imgrows}[1][\textwidth]
{\begin{minipage}{#1}
		\setcounter{row}{0}
		\stepcounter{figure}
	}
	{\addtocounter{figure}{-1}%
	\end{minipage}
}
\newcommand\imgrow
{\vspace{0.5em}\par\noindent
	\refstepcounter{row}
	\makebox[1.5em][r]{(\alph{row})}
}

%Numeración en líneas
\modulolinenumbers[1]
\setlength\parindent{0cm} 
\newpagestyle{main}{\setfoot{}{}{\thepage}}
\pagestyle{main}
\assignpagestyle{\chapter}{main}
\DeclareMathOperator*{\argmax}{arg\,max}
\DeclareMathOperator*{\argmin}{arg\,min}

%%%%%%%%%%%%%%%%%%%%%%%
%%%% Agregar información de carátula
%%%%%%%%%%%%%%%%%%%%%%%
\Facultad{DIRECCIÓN ZONAL AREQUIPA - PUNO}
\Escuela{ESCUELA/CFP: Tecnologías de la información}
\TProfesional{}

\title{\large{"Mejora en la presentación de servicios a través de la interactividad utilizando modelado 2D y animaciones."}}

\author{\large{Alex Diego Rosas Quispe}  \\ \large{Brandon Gonzales Tinta}}

\presidente{PRESIDENTE}{}
\primer{PRIMER JURADO}{}
\segundo{SEGUNDO JURAOD}{}
\asesor{TERCER JURADO}{}{}
\area{AREA}
\tema{TEMA}
%%%%%%%%%

%%%%%%%%%%%%%
% Acrónimos
%%%%%%%%%%%%%
\makeatletter
\newcommand{\acroforeign}[1]{}

% patch the environment to print the foreign definition:
\AtBeginEnvironment{acronym}{%
  \def\acroforeign#1{(#1)}%
}
% patch the acronym definition to safe the foreign definition:
\expandafter\patchcmd\csname AC@\AC@prefix{}@acro\endcsname
  {\begingroup}
  {\begingroup\def\acroforeign##1{\csdef{ac@#1@foreign}{##1, }}}
  {}
  {\fail}
% %   renew the first output to include the foreign definition if given:
\renewcommand*{\@acf}[2][\AC@linebreakpenalty]{%
  \ifAC@footnote
    \acsfont{\csname ac@#2@foreign\endcsname\AC@acs{#2}}%
    \footnote{\AC@placelabel{#2}\AC@acl{#2}{}}%
  \else
    \acffont{%
      \AC@placelabel{#2}\AC@acl{#2}%
      \nolinebreak[#1] %
      \acfsfont{(\acsfont{\csname ac@#2@foreign\endcsname\AC@acs{#2}})}%
    }%
  \fi
  \ifAC@starred\else\AC@logged{#2}\fi
}
\makeatother


%%%%%%% Quitar contador por defecto de TOF y TOT
\counterwithout{figure}{chapter}
\counterwithout{table}{chapter}

%%%%%%%%%%%%%%%%%%%%%%%%%%%
%indent according to section
\usepackage{changepage,lipsum,titlesec}
\titleformat{\section}[block]{\bfseries}{\thesection.}{1em}{}
\titleformat{\subsection}[block]{}{\thesubsection}{1em}{}
\titleformat{\subsubsection}[block]{}{\thesubsubsection}{1em}{}
\titlespacing*{\subsection} {2em}{3.25ex plus 1ex minus .2ex}{1.5ex plus .2ex}
\titlespacing*{\subsubsection} {3em}{3.25ex plus 1ex minus .2ex}{1.5ex plus .2ex}

%%%%%%%%%%%%%%%%%%%%%%%%%%%%%%%%%
%Tipo de letra: Times New Román en todo el texto del informe de tesis.
\usepackage{times}

%%%%%%%%%%%%%%%%%%%%%%%%%%%%%%%%%%%%%
%%%%%%%%%%%%%%%%%%%%%%%%%%%%%%%%%%%%%
%%%%%%%%%%%%%%%%%%%%%%%%%%%%%%%%%%%%%
%     COMIENZO DEL DOCUMENTO       %
%%%%%%%%%%%%%%%%%%%%%%%%%%%%%%%%%%%%%
%%%%%%%%%%%%%%%%%%%%%%%%%%%%%%%%%%%%%

\begin{document}

\captionsetup[table]{
  position=above,
  justification=raggedright,
  labelsep=newline, % <<< label and text on different lines
  singlelinecheck=false, % <<< raggadright also when the caption
  textfont=it
}

\captionsetup[figure]{
  justification=raggedright,
  singlelinecheck=false, % <<< raggadright also when the caption
  labelfont=it
}

\begin{spacing}{1.2}%espacio de la caratula
\maketitle %Compone la carátula 
\end{spacing}

\pagestyle{main} %Estilo

%%%%%%% Espacio y formato de títulos y subtítulos %%%%%%%%%%%
\makeatletter
\pretocmd{\@chapter}{% <--- IMPORTANT
\addtocontents{toc}{\cftpagenumbersoff{chapter}}% <--- IMPORTANT
}{}{}
\apptocmd{\@chapter}{% <--- IMPORTANT
\addtocontents{toc}{\cftpagenumberson{chapter}}% <--- IMPORTANT
}{}{}

\apptocmd{\@chapter}{% <--- IMPORTANT
 \addtocontents{toc}{\cftchappresnum\normalsize\bfseries{CAPÍTULO }}
 }{}{}
 \apptocmd{\@chapter}{% <--- IMPORTANT
 \addtocontents{toc}{\normalsize\bfseries\protect\centering\thechapter \\ \hspace{0.5em}#1\par}%
 }{}{}
\makeatother

%%%%%%% Espacio y formato de capítulo, secciones y subsecciones %%%%%%%%%%%

\renewcommand{\thechapter}{\Roman{chapter}}
\renewcommand{\thesection}{\arabic{chapter}.\arabic{section}}
\renewcommand{\theequation}{\arabic{equation}}

\titleformat{\chapter}[display]{\bfseries\centering}{\Large CAPÍTULO \thechapter}{0pt}{\Large #1}
%reglones titulo
\titlespacing{\chapter}{0pt}{-30pt}{0pt}

\titleformat{\section}
  {\normalfont\fontsize{12}{12}\bfseries}{\thesection}{1em}{#1}
\titleformat{\subsection}
  {\normalfont\fontsize{12}{12}\bfseries}{\thesubsection}{1em}{#1}
\titleformat{\subsubsection}
  {\normalfont\fontsize{12}{12}\bfseries}{\thesubsubsection}{1em}{#1}


%%%%%% INICIO DE INDICE GENERAL

\pagenumbering{roman}
\setcounter{page}{1} %Inserta pag al índice general
\addtocontents{toc}{~\hfill\textbf{Pág.}\par\medskip}
\begin{dedicatoria}

\hfill
\begin{flushright}
\hspace{8cm} \textit{Dedicatoria.}

\end{flushright}


\end{dedicatoria}

 %Inserta ldedicatoria
\addcontentsline{toc}{chapter}{\normalfont DEDICATORIA}
\begin{agradecimientos}


Agradecimientos

Quisiera expresar mi sincero agradecimiento a Ing. Rubén Valdemar Huanca Apaza 
por su invaluable orientación y apoyo durante todo el desarrollo de este proyecto.
Finalmente, mi gratitud a mi familia y amigos por su amor, paciencia y constante
estímulo

\end{agradecimientos}
 %Inserta los agradecimientos
\addcontentsline{toc}{chapter}{\normalfont AGRADECIMIENTOS}
\cleardoublepage
\renewcommand{\contentsname}{ÍNDICE GENERAL} %renombra  TOC a índice General
\phantomsection
\addcontentsline{toc}{chapter}{\normalfont ÍNDICE GENERAL}

%%%%%%%%%%%% ÍNDICE GENERAL
\tableofcontents %Inserta el índice general

%%%%%%%%%%%% ÍNDICE DE TABLAS
\cleardoublepage
\renewcommand{\listtablename}{ÍNDICE DE TABLAS}
\phantomsection
\addcontentsline{toc}{chapter}{\normalfont ÍNDICE DE TABLAS}
\addtocontents{lot}{~\hfill\textbf{Pág.}\par}
\setlength{\cfttabindent}{0pt}
\renewcommand{\cfttabpresnum}{\bfseries}
\renewcommand{\cfttabaftersnum}{.}
\listoftables

%%%%%%%%%%%% ÍNDICE DE FIGURAS
% \cleardoublepage
% \renewcommand{\listfigurename}{ÍNDICE DE FIGURAS}
% \phantomsection
% \addcontentsline{toc}{chapter}{\normalfont ÍNDICE DE FIGURAS}
% \addtocontents{lof}{~\hfill\textbf{Pág.}\par}
% \setlength{\cftfigindent}{0pt} %remove indent
% \renewcommand{\cftfigpresnum}{\bfseries}
% \renewcommand{\cftfigaftersnum}{.}
% \listoffigures %Inserta el índice de figuras
%

% %%%%%%%%%%%% ÍNDICE DE ANEXOS
% \cleardoublepage
% \phantomsection
% \addcontentsline{toc}{chapter}{\normalfont ÍNDICE DE ANEXOS}
% \addtocontents{apx}{~\hfill\textbf{Pág.}\par}
% \listofappendixchapter


%%%%%%%%%%%% ÍNDICE DE ACRÓNIMOS
% \cleardoublepage
% \phantomsection
% \addcontentsline{toc}{chapter}{\normalfont ÍNDICE DE ACRÓNIMOS}
% \renewcommand{\baselinestretch}{1.5}

\chapter*{ÍNDICE DE ACRÓNIMOS}

\begin{acronym}

\acro{CNN}{Red Neuronal Convolucional\acroforeign{Convolutional Neural Network}}

\end{acronym}



%%%%%%%%%%%% RESUMEN
\cleardoublepage
\phantomsection
\addcontentsline{toc}{chapter}{\normalfont RESUMEN}
\begin{resumen}

El resumen debe contener lo siguiente: objetivos, métodos empleados, resultados y conclusiones. La extensión máxima es de 250 palabras. La versión en español y la versión en inglés (abstract) tienen que decir lo mismo.

\textbf{Palabras Clave:} 

Las palabras clave (Keywords) son una lista de 5 a 8 palabras, relacionadas con las dimensiones del problema de investigación. Estas palabras se deben escribir después del resumen, en orden alfabético, separadas por comas y en minúsculas; excepto el inicio de los nombres propios. Las palabras clave se escriben también en inglés.

\end{resumen}

 %Inserta el resumen

%%%%%%%%%%%% ABSTRACT
% \cleardoublepage
% \phantomsection
% \addcontentsline{toc}{chapter}{\normalfont ABSTRACT}
% \begin{abstract}


\textbf{Keywords:} 


\end{abstract} %Inserta el abstract
%
%%%%%%%%%%%% INTRODUCCIÓN
% \cleardoublepage
% \pagenumbering{arabic}
% \setcounter{page}{1}
% \phantomsection
% \addcontentsline{toc}{chapter}{\normalfont INTRODUCCIÓN}
% \begin{introduccion}

Se elabora considerando los aspectos siguientes: el problema de investigación y su importancia, indicando el área, línea y tema de investigación de los programas de la Escuela de Posgrado; el propósito de investigación y los métodos. En un párrafo aparte, se debe exponer la estructura del informe de investigación.

\end{introduccion}


%%%%%%%%%%%%%%%%%%%%%%%%%%%%%%%%%%%%%%%%%%%%%%%%%%%%%%%%%%%%%%%%%%%%%
%%%%   En esta parte deberás incluir los archivos de tu tesis   %%%%%
%%%%%%%%%%%%%%%%%%%%%%%%%%%%%%%%%%%%%%%%%%%%%%%%%%%%%%%%%%%%%%%%%%%%%
\cleardoublepage
\pagestyle{main}

\authoredby{A}
\titlespacing{\chapter}{0pt}{120pt}{7pt}

\chapter{\large{Generalidades De La Empresa}}
\label{cap:generalidades}
\authoredby{B}

\section{Razón social}
\label{sec:razonsocial}

PROEFEX SOCIEDAD COMERCIAL DE RESPONSABILIDAD LIMITADA - PROEFEX S.R.L.

\section{Misión, Visión, Objetivos, Valores de la empresa.}

\subsection{Misión}

Aumentar el valor de nustros clientes de forma responsable, proponiendo y resolviendo con soluciones 
tecnológicas, innovadoras y creativas.

\subsection{Visión}

Ser una empresa referente en los rubros de tecnología e innovación y marketing en la región sur del Perú.

\subsection{Valores}

Agilidad, Responsabilidad

\section{Productos, mercado, clientes}
\label{sec:antecedentes} 

Esta sección contiene un mínimo de 20 estudios previos, que dan cuenta de los principales
hallazgos y contribuciones a la investigación. A partir de esta revisión de literatura sobre
el tema en estudio, se plantea el problema de investigación.
 %Inserta el capítulo 1
\authoredby{A}
\titlespacing{\chapter}{0pt}{120pt}{7pt}
\chapter{PLANTEAMIENTO DEL PROBLEMA}
\label{cap:problema}
\authoredby{B}

Esta sección contiene la definición del problema, la justificación, los objetivos y las hipótesis de investigación.
El planteamiento del problema y la postura epistémica deben ser claros, concisos, confiables y concordantes. En este sentido, expresa la relación entre el problema de investigación, la pregunta de investigación y la intención para investigar.


\section{Identificación del problema}

\section{Enunciados del problema}

\section{Justificación}
    
    En la justificación, se fundamentan la contribución a los vacíos de conocimiento y a la resolución del problema de investigación. Los objetivos de la investigación comprenden la intención del estudio y permiten describir el tipo de proceso que ha seguido la investigación. 
    
\section{Objetivos}
    \subsection{Objetivo general}
    \begin{adjustwidth}{2em}{0pt}
    Objetivo general
    \end{adjustwidth}
    
    
    \subsection{Objetivos específicos}
    \begin{adjustwidth}{2em}{0pt}
    \begin{itemize}
        \item Objetivo específico 1
        \item Objetivo específico 2
    \end{itemize}
    \end{adjustwidth}
    
    
\section{Hipótesis}

    Las hipótesis dan cuenta de las respuestas probables al problema objeto de investigación. %Inserta el capítulo 2
\authoredby{A}
\titlespacing{\chapter}{0pt}{120pt}{7pt}
\chapter{Análisis de la situación Actual}
\label{cap:metodologia}
\authoredby{B}

\section{Diagrama del proceso, mapa del flujo de valor y/o diagrama de operacón actual}

\textbf{Diagrama del proceso}

A continuación, se presenta un diagrama de flujo que ilustra el proceso de visita a la página web de Proefex. Este diagrama ofrece una representación visual del flujo de acciones que un usuario podría llevar a cabo al interactuar con el sitio web de la empresa.

El propósito de este diagrama es proporcionar una comprensión clara y concisa del recorrido típico de un visitante en el sitio web de Proefex, desde el inicio de la visita hasta el posible contacto con la empresa para obtener más información o servicios.

A través de esta representación gráfica, se busca destacar las etapas clave y las decisiones potenciales que un usuario puede enfrentar durante su interacción con la plataforma en línea de Proefex.

A continuación se muestra el diagrama de flujo:

\begin{figure}[!ht]
\centering
\tikzstyle{startstop} = [ellipse, draw, text width=3cm, text centered, minimum height=1cm]
\tikzstyle{process} = [rectangle, draw, text width=3cm, text centered, minimum height=1cm]
\tikzstyle{decision} = [diamond, draw, text width=3cm, text centered, minimum height=1cm]
\tikzstyle{arrow} = [-{Stealth[length=5mm,width=3mm]}, thick]

\begin{tikzpicture}[node distance=3.5cm, auto]
  \node [startstop] (start) {Inicio};
  \node [process, below of=start] (visit) {Visitar el sitio web de Proefex};
  \node [decision, below of=visit] (interested) {¿Interesado en servicios?};
  \node [process, below of=interested, yshift=-0.5cm] (explore) {Explorar servicios ofrecidos};
  \node [decision, below of=explore, yshift=-0.5cm] (contact) {¿Desea contactar?};
  \node [process, below of=contact, yshift=-0.5cm] (contactinfo) {Obtener información de contacto};
  \node [startstop, below of=contactinfo] (end) {Fin};

  \draw [arrow] (start) -- (visit);
  \draw [arrow] (visit) -- (interested);
  \draw [arrow] (interested) -- node[anchor=east] {Sí} (explore);
  \draw [arrow] (interested) -| node[anchor=south] {No}  ++(3,-3) |- (end);
  \draw [arrow] (explore) -- (contact);
  \draw [arrow] (contact) -- node[anchor=east] {Sí} (contactinfo);
  \draw [arrow] (contact) -| node[anchor=south] {No} ++(-3,-3) |- (end);
  \draw [arrow] (contactinfo) -- (end);
\end{tikzpicture}
\end{figure}

\newpage
\textbf{diagrama de operación}

El diagrama de operación presentado a continuación representa el flujo de interacción de un usuario al navegar por la página web de Proefex. Este diagrama se ha creado con el propósito de visualizar de manera clara y concisa las secciones principales a las que un usuario puede acceder durante su visita al sitio web de la empresa.

Las secciones representadas en el diagrama incluyen Inicio, Servicios, Eventos, Nosotros, Contacto, Blog y Recursos. Cada una de estas secciones desempeña un papel fundamental en la experiencia del usuario al proporcionar información, servicios, eventos, contacto con la empresa, contenido del blog y acceso a recursos adicionales respectivamente.

El flujo de operación mostrado en el diagrama refleja el recorrido esperado de un usuario al explorar las diversas secciones del sitio web de Proefex. Se busca destacar la navegación intuitiva y la conectividad entre las secciones, permitiendo al usuario moverse de manera fluida y lógica a través del contenido proporcionado por la empresa.

\begin{figure}[!ht]
\centering
\tikzstyle{circleblock} = [circle, draw, text width=1.5cm, text centered, minimum height=1.5cm]
\tikzstyle{squareblock} = [rectangle, draw, text width=1.5cm, text centered, minimum height=1.5cm]
\tikzstyle{line} = [-{Stealth[length=5mm,width=3mm]}, thick]

\begin{tikzpicture}[node distance=3cm, auto]
  \node [circleblock] (inicio) {Inicio};
  \node [circleblock, right of=inicio] (servicios) {Servicios};
  \node [circleblock, below of=servicios] (eventos) {Eventos};
  \node [circleblock, below of=inicio] (nosotros) {Nosotros};
  \node [circleblock, below of=nosotros] (contacto) {Contacto};
  \node [circleblock, below of=contacto] (blog) {Blog};
  \node [squareblock, right of=servicios, xshift=1.5cm] (recursos) {Recursos};

  \draw [line] (inicio) -- (servicios);
  \draw [line] (servicios) -- (eventos);
  \draw [line] (eventos) -- (nosotros);
  \draw [line] (nosotros) -- (contacto);
  \draw [line] (contacto) -- (blog);
  \draw [line] (blog) -- (recursos);
  \draw [line] (recursos) -- (servicios);
\end{tikzpicture}
\end{figure}

\newpage
\section{Efectos del problema en el área de trabajo o en los resultados de la empresa}

La limitada interacción y el enfoque estático en la presentación de servicios a 
través de texto e imágenes en la página web de Proefex pudo haber generado ciertos 
efectos en la experiencia del usuario y en los resultados de la empresa:

\begin{itemize}
\item \textbf{Limitación en la comprensión}

El enfoque estático puede haber dificultado la comprensión completa de los servicios
ofrecidos, ya que las descripciones visuales podrían no haber sido lo suficientemente
explícitas o atractivas.

\item \textbf{Posible Pérdida de Clientes Potenciales}

La falta de elementos interactivos y visuales atractivos podría haber reducido la
retención de visitantes y, por ende, la conversión de clientes potenciales.

\item \textbf{Menor Diferenciación de la Competencia}

Una presentación estática de servicios puede haber afectado la diferenciación
de Proefex respecto a sus competidores, limitando la percepción de innovación
y modernización.

\item \textbf{Menor Impacto de Marketing}

La falta de elementos visuales interactivos podría haber reducido el impacto del
marketing en línea, limitando el alcance y la efectividad de las campañas.

\end{itemize}

\section{Análisis de las causas raíces que generan el problema}

El problema identificado en la presentación estática de servicios en la página
web de Proefex puede tener múltiples causas subyacentes que han contribuido a
esta limitación en la interactividad y medios visuales dinámicos. Las posibles
causas raíces podrían ser:

\begin{itemize}
\item \textbf{Cultura Empresarial Conservadora}

Una mentalidad arraigada en presentaciones estáticas tradicionales, lo que ha
limitado la adopción de métodos más interactivos y visuales.

\item \textbf{Recursos Limitados para el Desarrollo}

Limitaciones presupuestarias o de recursos que han impedido la inversión en
herramientas o personal especializado en desarrollo interactivo.

\end{itemize}

\section{Priorización de causas raíces}

\begin{enumerate}
\item \textbf{Impacto en el Problema}
Evaluar la magnitud del efecto de cada causa raíz en la limitación de la interactividad
y presentación de servicios en la página web de Proefex.

\item \textbf{Frecuencia de Aparición}
Determinar con qué frecuencia o en qué medida cada causa raíz contribuye al problema.

\item \textbf{Factibilidad de Solución}
Evaluar la factibilidad de la solución actual para solucionar el problema.

\begin{table}[!ht]
\begin{center}
\begin{tabular}{| p{4cm} | p{3cm} | p{2cm} | p{2cm} | p{2cm} | }
\hline
Causa Raiz & Impacto en el Problema & Frecuencia & Factibilidad & Puntuación total \\ \hline
Cultura Empresarial Conservadora & Alta & Media & Alta & 8 \\ \hline
Recursos Limitados para el Desarrollo & Alta & Alta & Media & 9 \\ \hline
\end{tabular}

\caption{Priorización de causas raíces}
\label{tab:causasraiz}
\end{center}
\end{table} 
\end{enumerate}

\begin{figure}[ht]
  \centering
  \begin{tikzpicture}[
    level/.style={sibling distance=3cm/#1, level distance=5cm},
    grow=right
  ]
    \node {Recursos limitados}
      child {
        node {Peronal Escaso}
        child { node {Equipo Especializado en modelado} }
        child { node {Recursos Humanos} } 
      }
      child {
        node {Presupuesto Limitado}
        child { node {Importancia no tan relevante} }
        child { node {Tiempo Limitado} }
      };
  \end{tikzpicture}
  \caption{Sub-causas de Limitaciones Tecnológicas o de Recursos.}
\end{figure}

En esta sección, analizaremos de forma gráfica el problema mediante el Diagrama
de Pareto para identificar las áreas clave de enfoque. Este método nos permite
priorizar las limitaciones más significativas, optimizar recursos y diseñar
estrategias efectivas para mejorar la presentación de servicios mediante la
interactividad con modelos 2D y animaciones.

% Tabla de problemas y puntuaciones
\begin{table}[ht]
  \centering
  \caption{Descripción de Problemas y Puntuaciones}
  \begin{tabularx}{0.8\linewidth}{X c}
    \toprule
    \textbf{Problema} & \textbf{Puntuación} \\
    \midrule
    Falta de personal especializado en modelado & 80 \\
    Tiempo de inversión limitado & 65 \\
    Escasez de recursos económicos & 50 \\
    \bottomrule
  \end{tabularx}
\end{table}

% Tabla de causas y valoraciones
\begin{table}[ht]
  \centering
  \caption{Causas y Valoraciones}
  \begin{tabularx}{0.8\linewidth}{X c}
    \toprule
    \textbf{Causa} & \textbf{Valoración} \\
    \midrule
    Limitaciones de Recursos & 80 \\
    Limitaciones de Tiempo & 65 \\
    Restricciones Económicas & 50 \\
    \bottomrule
  \end{tabularx}
\end{table}



\begin{figure}[ht]
  \centering
  \begin{tikzpicture}
    \begin{axis}[
      ylabel=Puntuación,
      width=0.8\textwidth,
      height=6cm,
      ymin=0,
      ymax=200,
      xtick=data,
      symbolic x coords={A, B, C},
      xticklabels={A, B, C},
      nodes near coords,
      legend style={at={(0.5,-0.2)},anchor=north,legend columns=-1},
    ]
    \addplot[ybar,fill=blue] 
      coordinates {(A,80) (B,65) (C,50)};
    %
    \addplot[draw,mark=*] 
      coordinates {(A,80) (B,145) (C,195)};
    \legend{Puntuación, Curva de Pareto}
	% Leyenda en el gráfico
    \end{axis}
  \end{tikzpicture}
  \caption{Diagrama de Pareto con Curva de Distribución Acumulada.}
\end{figure}

\textbf{Leyenda:}
A: Limitaciones de Recursos \\
B: Limitaciones de Tiempo \\
C: Restricciones Económicas

\newpage

El resultado del Diagrama de Pareto nos muestra claramente las limitaciones más
relevantes que afectan la presentación de servicios con modelos 2D y animaciones
en Proefex. Al analizar este gráfico, se identifican las áreas prioritarias para
enfocar nuestros esfuerzos de mejora.

En términos de interpretación, las barras representan la magnitud de cada limitación
específica, permitiéndonos ver claramente cuáles tienen el mayor impacto. La curva
acumulada nos muestra la contribución acumulada de cada limitación, evidenciando 
las áreas que representan la mayor proporción del problema.

En este sentido, las limitaciones identificadas en los primeros lugares del 
gráfico (las barras más altas y la mayor contribución acumulada) son las áreas 
críticas que requieren atención inmediata para mejorar la presentación de
servicios. Esto nos guía en la asignación estratégica de recursos y esfuerzos
para abordar efectivamente los desafíos más relevantes.

 %Inserta el capítulo 3
\authoredby{A}
\titlespacing{\chapter}{0pt}{120pt}{7pt}
\chapter{RESULTADOS Y DISCUSIÓN}
\label{cap:resultados}
\authoredby{B}

Los resultados se presentan por objetivos específicos, desarrollando la interpretación de información contenida en tablas y/o figuras, demostrando la aceptación o rechazo de las hipótesis mediante la prueba estadística. No duplicar la presentación de resultados.

La forma de presentar es la siguiente: a) Interpretar los resultados, b) Citar en el texto, c) resultado estadístico, si existiera. y d) discusiones con otros autores. %Inserta el capítulo 4
\titlespacing{\chapter}{0pt}{120pt}{7pt}
\chapter{Costo de Implementación de la Mejora }
\label{cap:costo}
\authoredby{B}

\section{Costo de materiales}

En este apartado, se detallará el análisis de costos asociados a los materiales necesarios 
para la ejecución del proyecto de mejora. Este análisis resulta fundamental para 
comprender y cuantificar la inversión requerida en insumos específicos, permitiendo una 
gestión financiera precisa y una evaluación exhaustiva de la viabilidad económica del 
proyecto. La transparencia en la estimación de los costos de materiales es esencial para 
garantizar una planificación eficiente y el uso óptimo de recursos, maximizando así el 
impacto positivo de la implementación de las mejoras propuestas.

\begin{table}[!ht]
    \centering
    \begin{tabular}{|c|p{5cm}|c|c|c|}
    \hline
    \textbf{Item} & \textbf{Descripción} & \textbf{Cantidad} & \textbf{Costo Unitario} & \textbf{Monto Total} \\
    \hline
    1 & Licencia de software de modelado 2D -  3D & 1 & free & free \\
    \hline
    2 & Paquete de activos para animaciones & 1 & \$300 & \$300 \\
    \hline
    3 & Herramienta de desarrollo de juegos & 1 & \$700 & \$700 \\
    \hline
    4 & Suscripción a servicios en la nube & 1 & \$200 & \$200 \\
    \hline
    \end{tabular}
    \caption{Costos de Materiales para Mejorar Presentación de Servicios}
    \label{tab:costos_materiales_proefex}
\end{table}

\section{Costo de mano de obra}

El costo de mano de obra es un componente crucial en la implementación de mejoras de 
presentación de servicios de Proefex. La calidad y eficiencia en la ejecución del proyecto 
dependen en gran medida del personal involucrado. 

\begin{table}[htbp]
    \centering
    \begin{tabular}{|p{1cm}|p{2.5cm}|p{1.5cm}|p{1.5cm}|p{1.5cm}|p{1.5cm}|}
    \hline
    \textbf{Ítem} & \textbf{Descripción} & \textbf{Cantidad} & \textbf{Tarifa por Hora} & \textbf{Horas Estimadas} & \textbf{Monto Total} \\
    \hline
    1 & Desarrollador Frontend & 1 & \$30 & 100 & \$3000 \\
    \hline
    2 & Diseñador Gráfico & 1 & \$25 & 80 & \$2000 \\
    \hline
    3 & Especialista en Animación & 1 & \$35 & 120 & \$4200 \\
    \hline
    4 & Desarrollador Backend & 1 & \$30 & 100 & \$3000 \\
    \hline
    5 & Especialista en Modelado 2D - 3D & 1 & \$40 & 150 & \$6000 \\
    \hline
    6 & Gerente de Proyecto & 1 & \$50 & 80 & \$4000 \\
    \hline
    7 & QA/Tester & 1 & \$25 & 100 & \$2500 \\
    \hline
    8 & Soporte Técnico & 1 & \$20 & 60 & \$1200 \\
    \hline
    \multicolumn{5}{|r|}{\textbf{Total}} & \textbf{\$25900} \\
    \hline
    \end{tabular}
    \caption{Costo de Mano de Obra Estimado para Mejorar la Presentación de Servicios}
    \label{tab:costo_mano_de_obra}
\end{table}

\section{Costo de máquinas, herramientas y equipos}

\begin{table}[!ht]
    \centering
    \begin{tabular}{|p{3cm}|p{4cm}|p{2cm}|p{2cm}|p{2cm}|}
    \hline
    \textbf{Ítem} & \textbf{Descripción} & \textbf{Cantidad} & \textbf{Costo Unitario} & \textbf{Monto Total} \\
    \hline
    1 & Equipo de desarrollo de software & 1 & \$5000 & \$5000 \\
    \hline
    2 & Licencias de software especializado & 5 & \$1000 & \$5000 \\
    \hline
    3 & Equipamiento de oficina & 10 & \$500 & \$5000 \\
    \hline
    4 & Herramientas de diseño gráfico & 3 & \$1500 & \$4500 \\
    \hline
    \multicolumn{4}{|r|}{\textbf{Total}} & \textbf{\$19500} \\
    \hline
    \end{tabular}
    \caption{Costo de Máquinas, Herramientas y Equipos para el Desarrollo de una Página Web}
    \label{tab:costo_maquinas}
\end{table}

El costo asociado a las máquinas, herramientas y equipos es un factor crucial en la planificación de un proyecto como el desarrollo de una página web.

Esta tabla detalla los elementos esenciales necesarios para llevar a cabo dicho proyecto, desde el equipo de desarrollo de software hasta las licencias de software especializado y herramientas de diseño gráfico. La estimación de estos costos proporciona una visión detallada de los recursos necesarios para garantizar un entorno de trabajo óptimo y la implementación exitosa del proyecto. Así, esta información se convierte en un elemento clave en la gestión presupuestaria y en la toma de decisiones relacionadas con la inversión necesaria para llevar a cabo el desarrollo de la página web.

\section{Costo total de la implementación de la mejora}

El desglose de costos proporcionado refleja una evaluación detallada de los recursos necesarios para la implementación de la mejora en la presentación de servicios en la página web de la empresa Proefex. Estos costos consideran tanto los gastos asociados a la adquisición de equipos y herramientas especializadas, como los relacionados con la contratación de mano de obra cualificada y la obtención de materiales específicos para la ejecución del proyecto. El cálculo total representa la estimación financiera integral de la iniciativa, brindando una visión completa de la inversión requerida para llevar a cabo la mejora propuesta.

\begin{table}[!ht]
\centering
\begin{tabular}{|l|r|}
\hline
\textbf{Concepto} & \textbf{Costo Total} \\ \hline
Costo de máquinas, herramientas y equipos & \$19500 \\ \hline
Costo de mano de obra & \$25900 \\ \hline
Costo de materiales & \$1200 \\ \hline
\textbf{Total} & \textbf{\$46600} \\ \hline
\end{tabular}
\caption{Costo Total de Implementación de la Mejora}
\end{table}


\titlespacing{\chapter}{0pt}{120pt}{7pt}
\chapter{Evaluación Técnica y Económica de la Mejora}
\label{cap:evaluación}
\authoredby{B}

\section{Benefico técnico y/o econónico esperado de la mejora}

El beneficio técnico y económico esperado de la mejora en la presentación de servicios en la página web de Proefex se traduce en múltiples aspectos:

\textbf{Beneficio Técnico}

\begin{itemize}
\item \textbf{Mejora de Experiencia del Usuario:} La implementación de modelos 2D,
	animaciones interactivas y elementos tipo juego en la presentación de servicios
	en el sitio web mejorará la experiencia del usuario, haciéndola más atractiva y
	fácil de comprender.
\item \textbf{Aumento de Interacción:} La inclusión de elementos interactivos puede fomentar
la interacción del usuario con el contenido, lo que posiblemente aumente el tiempo
de permanencia en el sitio.
\item \textbf{Actualización Tecnológica:} La adopción de nuevas tecnologías en la presentación
de servicios coloca a la empresa en una posición más actualizada y competitiva en el
mercado.
\end{itemize}

\textbf{Beneficio Económico}

\begin{itemize}
\item \textbf{Atracción de Clientes Potenciales:} Una presentación más atractiva y dinámica
puede atraer a más clientes potenciales, lo que puede aumentar las conversiones y,
	  en última instancia, los ingresos.
\item \textbf{Retención de Clientes:} Una experiencia mejorada puede contribuir a la lealtad del
cliente y a una mayor retención, reduciendo posiblemente la tasa de rebote y aumentando
el retorno a largo plazo.
\item \textbf{Diferenciación Competitiva:} Al ofrecer una experiencia única y visualmente
impactante, la empresa puede destacar entre sus competidores, generando así un valor
adicional.
\end{itemize}

Estos beneficios, tanto en el ámbito técnico como económico, tienen el potencial de impactar positivamente en la percepción de la marca, el compromiso del usuario y, en última instancia, en el rendimiento financiero de la empresa.

\section{Relación Beneficio / Costo}

\cite{perez2023metodos} La relación Beneficio / Costo (B/C) se calcula mediante la fórmula:

\[
\text{B/C} = \frac{\text{Beneficio Total}}{\text{Costo Total}}
\]

\[
\text{B/C} = \frac{{19500 + 25900 + 1200}}{{19500 + 25900}} = \frac{46600}{45400} \approx 1.026
\]

Por tanto, la relación beneficio/costo calculada es aproximadamente 1.026, lo que indica que el proyecto tiene un rendimiento favorable en comparación con los costos incurridos. Esto sugiere una viabilidad positiva para la implementación de la mejora propuesta en términos de beneficios en relación con los costos asociados.


\titlespacing{\chapter}{0pt}{120pt}{7pt}
\chapter{Conclusiones}
\label{cap:concluciones}
\authoredby{B}

\section{Conclusiones respecto a los objetivos del Proyecto de Mejora}

El proyecto de mejora enfocado en la actualización de la presentación de servicios en la página web de Proefex ha logrado significativos avances. Se han implementado modelos 2D, animaciones interactivas y una experiencia de usuario similar a un juego, lo que ha permitido una visualización más dinámica y atractiva de los servicios ofrecidos.

A lo largo del desarrollo del proyecto, se ha observado un incremento en la interacción de los usuarios con la página, lo que sugiere un mayor interés y compromiso por parte de la audiencia. La adición de elementos interactivos ha generado una mayor retención de los visitantes, proporcionando una experiencia más inmersiva y efectiva.

Si bien los costos asociados con la implementación fueron considerables, el proyecto ha logrado generar un valor agregado notable para la empresa. La combinación de elementos visuales atractivos y una navegación más interactiva ha mejorado significativamente la presentación de los servicios, lo que probablemente resultará en un aumento de clientes potenciales y conversiones.

Es importante continuar monitoreando y analizando la recepción y el impacto de estas mejoras a largo plazo para evaluar su efectividad y justificar completamente la inversión realizada. Sin embargo, se espera que estas actualizaciones proporcionen una ventaja competitiva significativa para Proefex en su mercado objetivo.



\titlespacing{\chapter}{0pt}{120pt}{7pt}
\chapter{Recomendaciones}
\label{cap:recomendar}
\authoredby{B}

\section{Recomendaciones para la empresa respecto del Proyecto de Mejora}

Basado en los resultados obtenidos y la evolución del proyecto de mejora en la presentación de servicios en la página web, se sugieren las siguientes recomendaciones para Proefex:

1. Continuar el monitoreo: Mantener un seguimiento regular de las métricas clave de rendimiento web para evaluar la efectividad continua de las mejoras implementadas. Esto permitirá realizar ajustes y optimizaciones adicionales según sea necesario.

2. Recopilación de comentarios: Implementar sistemas de retroalimentación y encuestas para obtener comentarios directos de los usuarios sobre la nueva experiencia web. Esto proporcionará información valiosa para identificar áreas de mejora y satisfacer mejor las necesidades de los usuarios.

3. Optimización móvil: Asegurarse de que la experiencia del usuario en dispositivos móviles esté completamente optimizada, ya que un número significativo de usuarios acceden a la web a través de sus dispositivos móviles.

4. Capacitación del personal: Brindar capacitación continua al personal encargado de mantener y actualizar la página web. Esto garantizará que se utilicen todas las funciones y características implementadas de manera efectiva.

\newpage
5. Evolución constante: Estar al tanto de las últimas tendencias y tecnologías web para mantener la página actualizada y competitiva. La evolución constante es crucial para mantener el atractivo y la relevancia de la página web en un entorno digital cambiante.

Estas recomendaciones se centran en la optimización continua, el compromiso del usuario y la adaptación a las tendencias emergentes, lo que permitirá a Proefex mantener una presencia en línea sólida y atractiva.




% \cleardoublepage
% \phantomsection
% \addcontentsline{toc}{chapter}{\normalfont CONCLUSIONES}
% \titlespacing{\chapter}{0pt}{-30pt}{0pt}

\begin{conclusiones}
\label{cap:conclusiones}

Se desarrollan de acuerdo a los objetivos específicos y redactar en párrafos.

\end{conclusiones} %Inserta conclusiones

% \cleardoublepage
% \phantomsection
% \addcontentsline{toc}{chapter}{\normalfont RECOMENDACIONES}
% \titlespacing{\chapter}{0pt}{-30pt}{0pt}
\begin{recomendaciones}
\label{cap:recomendaciones}
 
Se presentan en relación con los objetivos específicos, dando cuenta de las consecuencias, orientaciones o medidas a realizarse.
    
\end{recomendaciones} %Inserta recomendaciones

%%%%%%%%%%%%%%%%%%%%%%%%%%%%%%%%%%%%%%%%%%%%%%%%%%%%%%%%%%%%%%%%%%%%%%
 \clearpage\phantomsection\addcontentsline{toc}{chapter}{\normalfont BIBLIOGRAFÍA}
 \renewcommand\bibname{BIBLIOGRAFÍA} %Inserta BIBLIOGRAFÍA EN APA7

\printbibliography 

% \bibliographystyle{apacite}
% \bibliography{mybib}


%%%%%%%%%%%
% \cleardoublepage
% \phantomsection
% \addcontentsline{toc}{chapter}{\normalfont ANEXOS}
% \begin{anexo}
\pagebreak
\hspace{0pt}
%\vfill
\newline
\newline
\newline
\newline
\newline
\begin{center}
\textbf{ANEXOS}
\end{center}
%\vfill
\hspace{0pt}
\pagebreak
\end{anexo} %Inserta Anexos
% \addtocontents{toc}{\protect\SpecialDocumentsTocHeading}

%%%%%%%%%%%%%
%CONFIGURACIÓN DE ANEXOS
%%%%%%%%%%%%%

% \appendix

\makeatletter
\def\@chapter[#1]#2{
                        \ifnum \c@secnumdepth >\m@ne
                         \refstepcounter{chapter}%
                         \typeout{\@chapapp\space\thechapter.}%
                         \addcontentsline{apx}{chapter}%NEW
                        {\protect\numberline{\thechapter} \normalfont#1}%
                       
                    \else
                      \addcontentsline{apx}{chapter}{#1}%NEW
                    \fi
                    \chaptermark{#1}%
                    \addtocontents{lof}{\protect\addvspace{10\p@}}%
                    \addtocontents{lot}{\protect\addvspace{10\p@}}%
                    \if@twocolumn
                    
                      \@topnewpage[\@makechapterhead{#2}]%
                    \else
                      \@makechapterhead{#2}%
                      \@afterheading
                    \fi}
\makeatother



\renewcommand*{\thechapter}{\bfseries\arabic{chapter}.}

\titleformat{\chapter}{\bfseries}{\large Anexo \thechapter}{0.5em}{\normalfont #1}

\titlespacing*{\chapter}{0cm}{-\topskip}{0pt}[0pt]

%%%%%%%%%%%%%
%AQUÍ PUEDES AGREGAR TUS ANEXOS
%%%%%%%%%%%%%


\end{document}
