%%%%%%%%%%%%%%%%%%%%%%%%%%%%%%%%%%%%%%%%%%%%%%%%%%%%%%%%%%%%%
% Plantilla para el Formato EPG cuantitativo de la 
% Escuela de POSGRADO DE LA UNIVERSIDAD NACIONAL DEL
% ALTIPLANO - PUNO. 
%
% La presente plantilla sigue las indicaciones del 
% siguiente enlace: https://drive.google.com/file/d/1yYM7YBnZbC7JnQ6O58tYuQ2SeFgz1ebW/view
% se recomienda revisar el formato base y la Resolución Directoral  
% Nº 0450-2020-DG-EPG-UNA-PUNO, para tener mayores detalles del 
% formato. 
%
% La plantilla utiliza el formato APA 7. Y esta dividido según los 
% capítulos solicitados por la Escuela de Posgrado-UNA-PUNO. Para editar
% la dedicatoria, agradecimientos y demás secciones y capítulos se debe 
% agregar el contenido en la pestaña correspondiente.
% Para añadir figuras se sugiere agregarlos en la carpeta gráficos.
% Para añadir referencias se realiza en mybib.bib
%
%Author Danitza Bermejo (danitza.bermejo@gmail.com)
%
%%%%%%%%%%%%%%%%%%%%%%%%%%%%%%%%%%%%%%%%%%%%%%%%%%%%%%%%%%%%%
% tamaño A4 , tamaño de fuente:12
\documentclass[a4paper,openany,12pt,oneside,hidelinks]{book}

% Tipo de letra: Times New Román en todo el texto del informe de tesis.
\usepackage[spanish,es-lcroman,es-tabla]{babel}
\usepackage{mathptmx}
\usepackage{graphicx,enumitem}
\usepackage[utf8]{inputenc}
\usepackage{ae}

%Márgenes: derecho, superior e inferior 2,5 cm. e izquierdo 3,5 cm
\usepackage[left=3.5cm,right=2.5cm,top=2.5cm,bottom=2.5cm]{geometry}

%Otras librerías utilizadas
\usepackage{acronym}
\usepackage{xspace}
\usepackage{hlundef}
\usepackage{tesis}
\usepackage{setspace}
\usepackage{lineno}
\usepackage{hyperref}
\usepackage{xpatch}
\usepackage{etoc}
\usepackage[utf8]{inputenc}
\usepackage[T1]{fontenc}

\newcommand*\SpecialDocumentsTocHeading
{\vspace{4ex}\parbox[t]{\textwidth}{.}\par
\global\let\SpecialDocumentsTocHeading\empty }

%Modificar Anexos
\usepackage{quotchap}
\usepackage[titles]{tocloft}
\renewcommand{\cftdot}{}
\renewcommand\cftchappagefont{\normalfont} %normal fond page
\usepackage{appendix}
\newlistof{appendixchapter}{apx}{ÍNDICE DE ANEXOS}

%%%%Para APA7th edition
 \usepackage{csquotes}
 \usepackage[style=apa,natbib=true,sortcites=true,sorting=nyt,backend=biber]{biblatex}
 \DefineBibliographyStrings{spanish}{andothers={\textit{et al}\adddot}}
\DeclareLanguageMapping{spanish}{spanish-apa}
\addbibresource{tesis.bib}

%%%% interlineado contenido
\usepackage[titles]{tocloft}
\setlength{\cftbeforechapskip}{3pt}

%%%tablas
\usepackage{color,soul}
\usepackage{etoolbox}
\usepackage{amsmath}
\usepackage{mathtools}
\usepackage{float} %H de las tablas
\usepackage{longtable} %tablas largas
\usepackage[spanish,onelanguage,ruled,vlined]{algorithm2e}

\usepackage[labelsep=period]{caption}
\usepackage{subcaption}
\usepackage{multicol}
\usepackage{multirow}
\usepackage{verbatim}
\usepackage[table,xcdraw]{xcolor} % para resaltar en las tablas
\usepackage{pdfpages} %añadir pdf
\usepackage{wrapfig} %para poner imagenes con texto
\usepackage{lscape}
\usepackage[explicit, pagestyles]{titlesec}
\usepackage{tabularx}
\usepackage{booktabs}
\newcolumntype{C}{>{\centering\arraybackslash}X} % centered version of "X" type
\setlength{\extrarowheight}{1pt}
\usepackage{makecell}
\newcommand{\ra}[1]{\renewcommand{\arraystretch}{#1}}
\usepackage{pifont} % DING
\usepackage{xcolor} % color 
\hypersetup{citecolor=blue}
\hypersetup{urlcolor=red}
\renewcommand\linenumberfont{\normalfont\small}

% interlineado del documento (1.5)
\renewcommand{\baselinestretch}{1.5}
\renewcommand{\arraystretch}{1.0}% interlineado tablas
\usepackage{multirow} %multi
\usepackage{verbatim} % comentarios
\usepackage{tikz}
\usetikzlibrary{positioning, shapes, shadows, arrows.meta}
\usepackage{pgfplots}
\usetikzlibrary{calc}
\usepackage{pgfgantt}
% \pgfplotsset{compat=1.17}

% BORRAR hipervínculo de referencias e hipervínculos de enlaces
\newcommand{\defineauthorcolor}[2]{%
  \colorlet{author#1}{#2}% Create an author colour
  \expandafter\def\csname authoredby#1\endcsname{% Create author colour settings
    \renewcommand{\cftchapfont}{\normalfont\color{author#1}}% Chapter colour
    \renewcommand{\cftchappagefont}{\normalfont\color{author#1}}% Chapter colour
    \renewcommand{\cftsecfont}{\normalfont\color{author#1}}% Section colour
    \renewcommand{\cftsubsecfont}{\normalfont\color{author#1}}}% Subsection colour
}
\makeatletter
\newcommand{\authoredby}[1]{\addtocontents{toc}{\protect\@nameuse{authoredby#1}}}%
\makeatother
\defineauthorcolor{A}{white}% Author A will be coloured red
\defineauthorcolor{B}{black}% Author B will be coloured blue
\def\UrlBreaks{\do\/\do-}

\newenvironment{figrow}%
{%
	\centering\addtocounter{figure}{1}% if caption at bottom
	\begin{enumerate}[%
		itemsep=2pt,parsep=0em,
		label={(\alph*)},
		ref={(\alph*)}]
}%
{\end{enumerate}\addtocounter{figure}{-1}}
\newcommand\img[1]
{\raisebox
	{\dimexpr-0.5\height+0.5ex}
	{\includegraphics[width=0.225\textwidth]{#1}}
}

%Ubicación texto en Figura
\newcounter{row}
\renewcommand\therow{\thefigure\alph{row}}
\newenvironment{imgrows}[1][\textwidth]
{\begin{minipage}{#1}
		\setcounter{row}{0}
		\stepcounter{figure}
	}
	{\addtocounter{figure}{-1}%
	\end{minipage}
}
\newcommand\imgrow
{\vspace{0.5em}\par\noindent
	\refstepcounter{row}
	\makebox[1.5em][r]{(\alph{row})}
}

%Numeración en líneas
\modulolinenumbers[1]
\setlength\parindent{0cm} 
\newpagestyle{main}{\setfoot{}{}{\thepage}}
\pagestyle{main}
\assignpagestyle{\chapter}{main}
\DeclareMathOperator*{\argmax}{arg\,max}
\DeclareMathOperator*{\argmin}{arg\,min}

%%%%%%%%%%%%%%%%%%%%%%%
%%%% Agregar información de carátula
%%%%%%%%%%%%%%%%%%%%%%%
\Facultad{DIRECCIÓN ZONAL AREQUIPA - PUNO}
\Escuela{ESCUELA/CFP: Tecnologías de la información}
\TProfesional{}

\title{\large{"Mejora en la presentación de servicios a través de la interactividad utilizando modelado 2D y animaciones."}}

\author{\large{Alex Diego Rosas Quispe}  \\ \large{Brandon Gonzales Tinta}}

\presidente{PRESIDENTE}{}
\primer{PRIMER JURADO}{}
\segundo{SEGUNDO JURAOD}{}
\asesor{TERCER JURADO}{}{}
\area{AREA}
\tema{TEMA}
%%%%%%%%%

%%%%%%%%%%%%%
% Acrónimos
%%%%%%%%%%%%%
\makeatletter
\newcommand{\acroforeign}[1]{}

% patch the environment to print the foreign definition:
\AtBeginEnvironment{acronym}{%
  \def\acroforeign#1{(#1)}%
}
% patch the acronym definition to safe the foreign definition:
\expandafter\patchcmd\csname AC@\AC@prefix{}@acro\endcsname
  {\begingroup}
  {\begingroup\def\acroforeign##1{\csdef{ac@#1@foreign}{##1, }}}
  {}
  {\fail}
% %   renew the first output to include the foreign definition if given:
\renewcommand*{\@acf}[2][\AC@linebreakpenalty]{%
  \ifAC@footnote
    \acsfont{\csname ac@#2@foreign\endcsname\AC@acs{#2}}%
    \footnote{\AC@placelabel{#2}\AC@acl{#2}{}}%
  \else
    \acffont{%
      \AC@placelabel{#2}\AC@acl{#2}%
      \nolinebreak[#1] %
      \acfsfont{(\acsfont{\csname ac@#2@foreign\endcsname\AC@acs{#2}})}%
    }%
  \fi
  \ifAC@starred\else\AC@logged{#2}\fi
}
\makeatother


%%%%%%% Quitar contador por defecto de TOF y TOT
\counterwithout{figure}{chapter}
\counterwithout{table}{chapter}

%%%%%%%%%%%%%%%%%%%%%%%%%%%
%indent according to section
\usepackage{changepage,lipsum,titlesec}
\titleformat{\section}[block]{\bfseries}{\thesection.}{1em}{}
\titleformat{\subsection}[block]{}{\thesubsection}{1em}{}
\titleformat{\subsubsection}[block]{}{\thesubsubsection}{1em}{}
\titlespacing*{\subsection} {2em}{3.25ex plus 1ex minus .2ex}{1.5ex plus .2ex}
\titlespacing*{\subsubsection} {3em}{3.25ex plus 1ex minus .2ex}{1.5ex plus .2ex}

%%%%%%%%%%%%%%%%%%%%%%%%%%%%%%%%%
%Tipo de letra: Times New Román en todo el texto del informe de tesis.
\usepackage{times}

%%%%%%%%%%%%%%%%%%%%%%%%%%%%%%%%%%%%%
%%%%%%%%%%%%%%%%%%%%%%%%%%%%%%%%%%%%%
%%%%%%%%%%%%%%%%%%%%%%%%%%%%%%%%%%%%%
%     COMIENZO DEL DOCUMENTO       %
%%%%%%%%%%%%%%%%%%%%%%%%%%%%%%%%%%%%%
%%%%%%%%%%%%%%%%%%%%%%%%%%%%%%%%%%%%%

\begin{document}

\captionsetup[table]{
  position=above,
  justification=raggedright,
  labelsep=newline, % <<< label and text on different lines
  singlelinecheck=false, % <<< raggadright also when the caption
  textfont=it
}

\captionsetup[figure]{
  justification=raggedright,
  singlelinecheck=false, % <<< raggadright also when the caption
  labelfont=it
}

\begin{spacing}{1.2}%espacio de la caratula
\maketitle %Compone la carátula 
\end{spacing}

\pagestyle{main} %Estilo

%%%%%%% Espacio y formato de títulos y subtítulos %%%%%%%%%%%
\makeatletter
\pretocmd{\@chapter}{% <--- IMPORTANT
\addtocontents{toc}{\cftpagenumbersoff{chapter}}% <--- IMPORTANT
}{}{}
\apptocmd{\@chapter}{% <--- IMPORTANT
\addtocontents{toc}{\cftpagenumberson{chapter}}% <--- IMPORTANT
}{}{}

\apptocmd{\@chapter}{% <--- IMPORTANT
 \addtocontents{toc}{\cftchappresnum\normalsize\bfseries{CAPÍTULO }}
 }{}{}
 \apptocmd{\@chapter}{% <--- IMPORTANT
 \addtocontents{toc}{\normalsize\bfseries\protect\centering\thechapter \\ \hspace{0.5em}#1\par}%
 }{}{}
\makeatother

%%%%%%% Espacio y formato de capítulo, secciones y subsecciones %%%%%%%%%%%

\renewcommand{\thechapter}{\Roman{chapter}}
\renewcommand{\thesection}{\arabic{chapter}.\arabic{section}}
\renewcommand{\theequation}{\arabic{equation}}

\titleformat{\chapter}[display]{\bfseries\centering}{\Large CAPÍTULO \thechapter}{0pt}{\Large #1}
%reglones titulo
\titlespacing{\chapter}{0pt}{-30pt}{0pt}

\titleformat{\section}
  {\normalfont\fontsize{12}{12}\bfseries}{\thesection}{1em}{#1}
\titleformat{\subsection}
  {\normalfont\fontsize{12}{12}\bfseries}{\thesubsection}{1em}{#1}
\titleformat{\subsubsection}
  {\normalfont\fontsize{12}{12}\bfseries}{\thesubsubsection}{1em}{#1}


%%%%%% INICIO DE INDICE GENERAL

\pagenumbering{arabic}
\setcounter{page}{1} %Inserta pag al índice general
\addtocontents{toc}{~\hfill\textbf{Pág.}\par\medskip}
\begin{dedicatoria}

\hfill
\begin{flushright}
\hspace{8cm} \textit{Dedicatoria.}

A mis padres, por su apoyo incondicional y por ser mi inspiración.

A mis compañeros de equipo, por su colaboración y esfuerzo. 

Este proyecto está dedicado a todos aquellos que creen en la innovación y el trabajo arduo para mejorar nuestro entorno.

Gracias.
\end{flushright}


\end{dedicatoria}

 %Inserta ldedicatoria
\addcontentsline{toc}{chapter}{\normalfont DEDICATORIA}
\begin{agradecimientos}


Agradecimientos

Quisiera expresar mi sincero agradecimiento a Ing. Rubén Valdemar Huanca Apaza 
por su invaluable orientación y apoyo durante todo el desarrollo de este proyecto.
Finalmente, mi gratitud a mi familia y amigos por su amor, paciencia y constante
estímulo

\end{agradecimientos}
 %Inserta los agradecimientos
\addcontentsline{toc}{chapter}{\normalfont AGRADECIMIENTOS}
\cleardoublepage
\renewcommand{\contentsname}{ÍNDICE GENERAL} %renombra  TOC a índice General
\phantomsection
\addcontentsline{toc}{chapter}{\normalfont ÍNDICE GENERAL}

%%%%%%%%%%%% ÍNDICE GENERAL
\tableofcontents %Inserta el índice general

%%%%%%%%%%%% ÍNDICE DE TABLAS
\cleardoublepage
\renewcommand{\listtablename}{ÍNDICE DE TABLAS}
\phantomsection
\addcontentsline{toc}{chapter}{\normalfont ÍNDICE DE TABLAS}
\addtocontents{lot}{~\hfill\textbf{Pág.}\par}
\setlength{\cfttabindent}{0pt}
\renewcommand{\cfttabpresnum}{\bfseries}
\renewcommand{\cfttabaftersnum}{.}
\listoftables

%%%%%%%%%%%% ÍNDICE DE FIGURAS
% \cleardoublepage
% \renewcommand{\listfigurename}{ÍNDICE DE FIGURAS}
% \phantomsection
% \addcontentsline{toc}{chapter}{\normalfont ÍNDICE DE FIGURAS}
% \addtocontents{lof}{~\hfill\textbf{Pág.}\par}
% \setlength{\cftfigindent}{0pt} %remove indent
% \renewcommand{\cftfigpresnum}{\bfseries}
% \renewcommand{\cftfigaftersnum}{.}
% \listoffigures %Inserta el índice de figuras
%

% %%%%%%%%%%%% ÍNDICE DE ANEXOS
% \cleardoublepage
% \phantomsection
% \addcontentsline{toc}{chapter}{\normalfont ÍNDICE DE ANEXOS}
% \addtocontents{apx}{~\hfill\textbf{Pág.}\par}
% \listofappendixchapter


%%%%%%%%%%%% ÍNDICE DE ACRÓNIMOS
% \cleardoublepage
% \phantomsection
% \addcontentsline{toc}{chapter}{\normalfont ÍNDICE DE ACRÓNIMOS}
% \renewcommand{\baselinestretch}{1.5}

\chapter*{ÍNDICE DE ACRÓNIMOS}

\begin{acronym}

\acro{CNN}{Red Neuronal Convolucional\acroforeign{Convolutional Neural Network}}

\end{acronym}



%%%%%%%%%%%% RESUMEN
\cleardoublepage
\phantomsection
\addcontentsline{toc}{chapter}{\normalfont RESUMEN}
\begin{resumen}

\textbf{Objetivos del Proyecto:}

El proyecto se centra en mejorar la presentación de servicios en la página web de Proefex mediante la integración de modelos 2D, animaciones interactivas y elementos de juego. Los objetivos específicos incluyen enriquecer la experiencia del usuario, aumentar la interactividad, y mejorar la visualización de los servicios ofrecidos por la empresa.

\textbf{Antecedentes:}

El entorno digital actual exige presentaciones más dinámicas y atractivas para captar y retener la atención del usuario. En respuesta a esta demanda, se identificó la necesidad de renovar la presentación de servicios en la página web de Proefex para mantenerse actualizados y competitivos en el mercado.

\textbf{Análisis de la Mejora:}

La implementación de modelos 2D, animaciones interactivas y elementos de juego proporcionará una experiencia más atractiva y envolvente para los visitantes del sitio web. Esto ayudará a destacar los servicios ofrecidos, facilitar la comprensión de la información, y aumentar el tiempo de permanencia en la página, lo que potencialmente impulsará las conversiones y la retención de usuarios.

\textbf{Plan Propuesto:}

El proyecto se dividirá en fases, que incluyen el diseño conceptual, el desarrollo técnico de modelos y animaciones, la integración de elementos de juego, y la fase de pruebas para garantizar la funcionalidad y la experiencia del usuario. Se asignará un equipo multidisciplinario para llevar a cabo estas tareas de manera eficiente y coordinada.

\textbf{Resultados Económicos:}

Si bien la inversión inicial en el proyecto puede ser considerable, se espera que los beneficios a largo plazo superen los costos. La mejora en la presentación de servicios puede atraer a un mayor número de clientes potenciales, aumentar las conversiones y la retención de usuarios, lo que en última instancia puede traducirse en un crecimiento económico sostenido para Proefex.

Este proyecto busca no solo modernizar la presentación de servicios en el sitio web de Proefex, sino también mejorar la experiencia del usuario y potenciar el rendimiento económico de la empresa en un entorno digital competitivo.
\end{resumen}

 %Inserta el resumen

%%%%%%%%%%%% ABSTRACT
% \cleardoublepage
% \phantomsection
% \addcontentsline{toc}{chapter}{\normalfont ABSTRACT}
% \begin{abstract}


\textbf{Keywords:} 


\end{abstract} %Inserta el abstract
%
%%%%%%%%%%%% INTRODUCCIÓN
% \cleardoublepage
% \pagenumbering{arabic}
% \setcounter{page}{1}
% \phantomsection
% \addcontentsline{toc}{chapter}{\normalfont INTRODUCCIÓN}
% \begin{introduccion}

Se elabora considerando los aspectos siguientes: el problema de investigación y su importancia, indicando el área, línea y tema de investigación de los programas de la Escuela de Posgrado; el propósito de investigación y los métodos. En un párrafo aparte, se debe exponer la estructura del informe de investigación.

\end{introduccion}


%%%%%%%%%%%%%%%%%%%%%%%%%%%%%%%%%%%%%%%%%%%%%%%%%%%%%%%%%%%%%%%%%%%%%
%%%%   En esta parte deberás incluir los archivos de tu tesis   %%%%%
%%%%%%%%%%%%%%%%%%%%%%%%%%%%%%%%%%%%%%%%%%%%%%%%%%%%%%%%%%%%%%%%%%%%%
\cleardoublepage
\pagestyle{main}

\authoredby{A}
\titlespacing{\chapter}{0pt}{120pt}{7pt}

\chapter{\large{Generalidades De La Empresa}}
\label{cap:generalidades}
\authoredby{B}

\section{Razón social}
\label{sec:razonsocial}

PROEFEX SOCIEDAD COMERCIAL DE RESPONSABILIDAD LIMITADA - PROEFEX S.R.L.

\section{Misión, Visión, Objetivos, Valores de la empresa.}

\subsection{Misión}

Aumentar el valor de nustros clientes de forma responsable, proponiendo y resolviendo con soluciones 
tecnológicas, innovadoras y creativas.

\subsection{Visión}

Ser una empresa referente en los rubros de tecnología e innovación y marketing en la región sur del Perú.

\subsection{Valores}

Agilidad, Responsabilidad

\section{Productos, mercado, clientes}
\label{sec:antecedentes} 

Esta sección contiene un mínimo de 20 estudios previos, que dan cuenta de los principales
hallazgos y contribuciones a la investigación. A partir de esta revisión de literatura sobre
el tema en estudio, se plantea el problema de investigación.
 %Inserta el capítulo 1
\authoredby{A}
\titlespacing{\chapter}{0pt}{120pt}{7pt}
\chapter{Plan Del Proyecto de Innovación y/o mejora}
\label{cap:problema}
\authoredby{B}

\section{Identificación del problema técnico en la empresa}

Para abordar la identificación del problema, optaremos por emplear la metodología de lluvia
de ideas. A través de esta técnica, recopilaremos ideas y sugerencias provenientes de distintas
perspectivas y experiencias. Posteriormente, analizaremos los resultados obtenidos en las tablas
pertinentes para determinar el problema que requerirá nuestra atención y resolución.

\begin{table}[!ht]
\begin{center}
\begin{tabular}{| c | p{10cm} |}
\hline
\multicolumn{2}{ |c| }{Alex Diego Rosas Quispe} \\ \hline
Item & Problema \\ \hline
1 & Limitada interactividad en la página web actual \\ \hline
2 & Información poco visual y atractiva para los visitantes \\ \hline
3 & Falta de elementos dinámicos para resaltar los servicios ofrecidos \\ \hline
\end{tabular}
\caption{Ideas propuestas por Alex Diego Rosas Quispe}
\label{tab:ideasalex}
\end{center}
\end{table}

\newpage

\begin{table}[!ht]
\begin{center}
\begin{tabular}{| c | p{10cm} |}
\hline
\multicolumn{2}{ |c| }{Brandon Gonzales Tinta} \\ \hline
Item & Problema \\ \hline
1 & Experiencia de usuario desactualizada y poco atractiva \\ \hline
2 & Falta de contenido interactivo para retener la atención del visitante \\ \hline
3 & Ausencia de elementos visuales atractivos para los servicios ofrecidos \\ \hline
\end{tabular}
\caption{Ideas propuestas por Brandon Gonzalez Tinta}
\label{tab:ideasbrandon}
\end{center}
\end{table}

\begin{table}[!ht]
\begin{center}
\begin{tabular}{| p{4cm} | p{5cm} | p{5cm} |}
\hline
\multicolumn{3}{|p{4cm}|}{Tabla de Afinidades} \\ \hline
Ideas Base & \multicolumn{2}{|p {10 cm}|}{Ideas Planteadas} \\ \hline
Limitada interactividad en la página web actual & Experiencia de usuario desactualizada y poco atractiva & Falta de contenido interactivo para retener la atención del visitante \\ \hline
Información poco visual y atractiva para los visitantes & Ausencia de elementos visuales atractivos para los servicios ofrecidos & \\ \hline
Falta de elementos dinámicos para resaltar los servicios ofrecidos & & \\ \hline
\end{tabular}
\caption{Tabla de Afinidades entre Ideas Base e Ideas Planteadas}
\label{tab:afinidades}
\end{center}
\end{table}


\begin{table}[!ht]
\begin{center}
\begin{tabular}{|p{5cm}|c|c|c|}
\hline
\textbf{IDEAS BASE} & \textbf{FRECUENCIA} & \textbf{IMPORTANCIA} & \textbf{FACTIBILIDAD} \\ \hline
Limitada interactividad en la página web actual & 3 & 4 & 3 \\ \hline
Información poco visual y atractiva para los visitantes & 2 & 5 & 4 \\ \hline
Falta de elementos dinámicos para resaltar los servicios ofrecidos & 4 & 3 & 2 \\ \hline
\end{tabular}
\caption{Tabla de Priorización de Ideas Base}
\label{tab:priorizacion}
\end{center}
\end{table}

El análisis de afinidades y priorización de ideas es una parte crucial en la identificación y resolución de problemas. Estas tablas permiten una evaluación estructurada y meticulosa de varias ideas planteadas por diferentes personas, lo que facilita la identificación de patrones comunes, la frecuencia de ocurrencia de ciertos problemas, la importancia percibida de cada problema y su factibilidad para su resolución.

La tabla de afinidades agrupa ideas similares o relacionadas, permitiendo identificar áreas clave que necesitan atención. Por otro lado, la tabla de priorización ayuda a clasificar estas ideas según su frecuencia, importancia y viabilidad, lo que permite enfocarse en las áreas más críticas y determinar qué problemas pueden ser abordados de manera más efectiva y con mayor impacto.

Estos procesos estructurados no solo ayudan a organizar ideas, sino que también brindan un marco para la toma de decisiones más informada y estratégica al priorizar los problemas que requieren atención inmediata y aquellos que pueden abordarse en etapas posteriores.

\section{Objetivos del Proyecto de Innovación y/o Mejora}

\textbf{Objetivo General}

Optimizar la presentación de los servicios a través de la implementación de elementos interactivos,
modelado 2D y animaciones, con el propósito de proporcionar una experiencia más atractiva,
informativa y memorable para los clientes, fortaleciendo así la comunicación efectiva de los
valores y beneficios de los servicios.

\textbf{Objetios Especificos}

\begin{itemize}
\item Crear presentaciones interactivas que involucren a los usuarios. 
\item Utilizar modelado 2D para representar de manera precisa y atractiva los aspectos clave de los servicios ofrecidos. 
\end{itemize}

\section{Antecedentes del Proyecto de Innovación y/o mejora (Investigaciones realizadas)}

\cite{lizarraga2014blended} Blended-learning afectivo y las herramientas interactivas de la Web 3.0: una revisión sistemática de la literatura.

\cite{cordova2017turismo} Turismo, web 2.0 y Comunicación Interactiva en América Latina. Buenas prácticas y tendencias.

\cite{santos2023interactividad} Interactividad, buscabilidad y visibilidad web en periodismo digital galardonado.
   
\section{Justificación del Proyecto de Innovación y/o Mejora}
La presente propuesta de mejora busca fortalecer la presentación de servicios de Proefex a
través de la implementación de herramientas interactivas basadas en modelado 2D y
animaciones. En un mundo cada vez más digitalizado, la capacidad de presentar los
servicios de manera dinámica y atractiva se ha convertido en un factor crucial para
captar la atención y generar un impacto significativo en el mercado. 

Con el fin de mantenerse a la vanguardia en la industria, es fundamental adaptarse a las
demandas y expectativas de los clientes actuales, quienes valoran la interactividad y la
visualización dinámica como medios efectivos para comprender los servicios ofrecidos.
Mediante el uso de modelado 2D y animaciones, Proefex podrá potenciar la presentación de
sus servicios, ofreciendo una experiencia más inmersiva y atractiva para sus potenciales
clientes.

Además, la implementación de estas herramientas no solo mejorará la presentación de
servicios, sino que también permitirá destacar la innovación y el compromiso de Proefex
con la excelencia en la prestación de servicios, consolidando su posición como líder en
el mercado de Arequipa, Perú. Esta iniciativa no solo elevará la percepción de la
empresa, sino que también contribuirá a aumentar la visibilidad y atraer nuevos
clientes, fortaleciendo así su posición competitiva en el sector. 

En resumen, esta propuesta de mejora se fundamenta en la necesidad de adaptación a un
entorno empresarial cada vez más orientado hacia la interactividad y la presentación
dinámica de servicios, permitiendo a Proefex elevar su oferta, mejorar su imagen de
marca y mantener su competitividad en el mercado local y regional.

\section{Marco Teorico y Conceptual}

\subsection{Fundamento teórico del Proyecto de Innovación y Mejora}

 Interactividad y Experiencia del Usuario (UX/UI): La interactividad juega un papel
fundamental en la experiencia del usuario. Teorías de diseño centradas en el usuario,
como la Teoría de la Usabilidad de Nielsen, destacan la importancia de interfaces
interactivas para atraer, retener y comprometer a los usuarios.

 Modelado 2D y Animaciones: En el ámbito del diseño y la presentación, el modelado 2D
 y las animaciones se basan en principios de diseño visual, teorías de percepción y
 psicología del color y la forma. Conceptos como la Ley de la Continuidad Gestáltica
 y la teoría del movimiento en animación respaldan la efectividad de estas técnicas
 para captar la atención y transmitir información de manera efectiva.

 Aprendizaje visual y Memoria: La teoría del aprendizaje visual sostiene que las
 personas tienden a recordar mejor la información cuando se presenta de manera visual
 y dinámica. Esto se relaciona con la teoría de la memoria cognitiva, que sugiere que
 la información visual se procesa y retiene de manera más eficiente que la información
 puramente textual.

 Marketing y Comunicación Visual: Teorías de marketing como el Marketing Sensorial
 respaldan la importancia de estimular los sentidos y crear experiencias memorables
 para influir en las decisiones de compra. La teoría de la Comunicación Visual subraya
 cómo los elementos visuales impactan la percepción y comprensión del público objetivo.

  Tecnología y Tendencias Digitales: La rápida evolución tecnológica y las tendencias
  digitales actuales respaldan la implementación de herramientas interactivas y
  animaciones como una estrategia efectiva para destacar y diferenciar los servicios
  ofrecidos en un mercado cada vez más competitivo.

  La combinación de estos fundamentos teóricos respalda la validez y eficacia del uso
  de modelado 2D y animaciones para mejorar la presentación de servicios, ofreciendo
  una experiencia más atractiva, memorable y efectiva para los clientes potenciales de
  Proefex.

\subsection{Conceptos y términos utilizados}

\begin{itemize}
    \item Interactividad: Capacidad de los usuarios para interactuar con una interfaz, sistema o contenido digital, permitiendo acciones y respuestas bidireccionales.
    \item Modelado 2D: Proceso de crear representaciones bidimensionales de objetos, entornos o diseños utilizando software especializado.
    \item Animaciones: Secuencias de imágenes en movimiento creadas mediante la manipulación y reproducción de una serie de cuadros estáticos.
    \item Experiencia del Usuario (UX): Enfoque en el diseño de productos o servicios centrado en la satisfacción y facilidad de uso percibida por los usuarios al interactuar con ellos. \cite{ferrer2023aplicabilidad}
    \item Diseño Visual: Aplicación de principios y técnicas visuales para comunicar ideas o conceptos de manera efectiva y atractiva.
    \item Usabilidad: Grado en el que un producto o sistema puede ser utilizado por usuarios específicos para alcanzar objetivos específicos con eficacia, eficiencia y satisfacción en un contexto específico de uso.
    \item Teoría de la Usabilidad (Nielsen): Marco teórico desarrollado por Jakob Nielsen que establece principios y pautas para mejorar la usabilidad de los productos digitales. \cite{preciado2023analisis} 
    \item Ley de la Continuidad Gestáltica: Principio de percepción visual que postula que los elementos visuales tienden a ser percibidos de manera continua cuando se alinean o continúan en una dirección específica. \cite{ciafardo2020breviario}
    \item Psicología del Color: Estudio de cómo los colores afectan la percepción y el comportamiento humano. \cite{garcia2023psicologia}
    \item Teoría del Aprendizaje Visual: Concepto que sugiere que las personas retienen mejor la información cuando se presenta visualmente en lugar de solo texto. \cite{penaherrera2023paratextualidad}
    \item Marketing Sensorial: Estrategia de marketing que busca estimular los sentidos del consumidor para influir en sus emociones y decisiones de compra.
    \item Comunicación Visual: Uso de elementos visuales para transmitir información, ideas o mensajes de manera efectiva.
    \item Tecnologías Emergentes: Nuevas tecnologías que están surgiendo o ganando prominencia en un campo específico.
\end{itemize}

 %Inserta el capítulo 2
\authoredby{A}
\titlespacing{\chapter}{0pt}{120pt}{7pt}
\chapter{MATERIALES Y MÉTODOS}
\label{cap:metodologia}
\authoredby{B}

\section{Lugar de estudio}

Se debe identificar el lugar donde se realizó la investigación y la georreferencia. Así como, las características ambientales, socioeconómicas y culturales. Sustentar la importancia de la zona de estudio y de sus actores.

\section{Población}

La población es un grupo de sujetos u objetos con unas características definitorias diversas. El investigador de acuerdo a los objetivos define los criterios de inclusión y exclusión.

\section{Muestra}

La muestra es representativa de la población. Por ello, se describe la técnica de muestreo adecuada.

\section{Método de Investigación}

El investigador explica el método de investigación según los objetivos y las variables de la investigación: descriptiva, explicativa o experimental.


\section{Descripción detallada de métodos por objetivos específicos}

Considerar en la presentación de la metodología: a) Descripción de variables analizadas en los objetivos específicos, b) Descripción detallada del uso de materiales, equipos, instrumentos, insumos, entre otros y c) Aplicación de prueba estadística inferencial. %Inserta el capítulo 3
\authoredby{A}
\titlespacing{\chapter}{0pt}{120pt}{7pt}
\chapter{RESULTADOS Y DISCUSIÓN}
\label{cap:resultados}
\authoredby{B}

Los resultados se presentan por objetivos específicos, desarrollando la interpretación de información contenida en tablas y/o figuras, demostrando la aceptación o rechazo de las hipótesis mediante la prueba estadística. No duplicar la presentación de resultados.

La forma de presentar es la siguiente: a) Interpretar los resultados, b) Citar en el texto, c) resultado estadístico, si existiera. y d) discusiones con otros autores. %Inserta el capítulo 4
\titlespacing{\chapter}{0pt}{120pt}{7pt}
\chapter{Costo de Implementación de la Mejora }
\label{cap:costo}
\authoredby{B}

\section{Costo de materiales}

En este apartado, se detallará el análisis de costos asociados a los materiales necesarios 
para la ejecución del proyecto de mejora. Este análisis resulta fundamental para 
comprender y cuantificar la inversión requerida en insumos específicos, permitiendo una 
gestión financiera precisa y una evaluación exhaustiva de la viabilidad económica del 
proyecto. La transparencia en la estimación de los costos de materiales es esencial para 
garantizar una planificación eficiente y el uso óptimo de recursos, maximizando así el 
impacto positivo de la implementación de las mejoras propuestas.

\begin{table}[!ht]
    \centering
    \begin{tabular}{|c|c|c|c|c|}
    \hline
    \textbf{Item} & \textbf{Descripción} & \textbf{Cantidad} & \textbf{Costo Unitario} & \textbf{Monto Total} \\
    \hline
    1 & Licencia de software de modelado 2D -  3D & 1 & free & free \\
    \hline
    2 & Paquete de activos para animaciones & 1 & \$300 & \$300 \\
    \hline
    3 & Herramienta de desarrollo de juegos & 1 & \$700 & \$700 \\
    \hline
    4 & Suscripción a servicios en la nube & 1 & \$200 & \$200 \\
    \hline
    \end{tabular}
    \caption{Costos de Materiales para Mejorar Presentación de Servicios}
    \label{tab:costos_materiales_proefex}
\end{table}

\section{Costo de mano de obra}

El costo de mano de obra es un componente crucial en la implementación de mejoras de 
presentación de servicios de Proefex. La calidad y eficiencia en la ejecución del proyecto 
dependen en gran medida del personal involucrado. 

\newpage

\begin{table}[htbp]
    \centering
    \begin{tabular}{|p{1cm}|p{2.5cm}|p{1.5cm}|p{1.5cm}|p{1.5cm}|p{1.5cm}|}
    \hline
    \textbf{Ítem} & \textbf{Descripción} & \textbf{Cantidad} & \textbf{Tarifa por Hora} & \textbf{Horas Estimadas} & \textbf{Monto Total} \\
    \hline
    1 & Desarrollador Frontend & 1 & \$30 & 100 & \$3000 \\
    \hline
    2 & Diseñador Gráfico & 1 & \$25 & 80 & \$2000 \\
    \hline
    3 & Especialista en Animación & 1 & \$35 & 120 & \$4200 \\
    \hline
    4 & Desarrollador Backend & 1 & \$30 & 100 & \$3000 \\
    \hline
    5 & Especialista en Modelado 2D - 3D & 1 & \$40 & 150 & \$6000 \\
    \hline
    6 & Gerente de Proyecto & 1 & \$50 & 80 & \$4000 \\
    \hline
    7 & QA/Tester & 1 & \$25 & 100 & \$2500 \\
    \hline
    8 & Soporte Técnico & 1 & \$20 & 60 & \$1200 \\
    \hline
    \multicolumn{5}{|r|}{\textbf{Total}} & \textbf{\$25900} \\
    \hline
    \end{tabular}
    \caption{Costo de Mano de Obra Estimado para Mejorar la Presentación de Servicios}
    \label{tab:costo_mano_de_obra}
\end{table}

\section{Costo de máquinas, herramientas y equipos}

\begin{table}[!ht]
    \centering
    \begin{tabular}{|p{3cm}|p{4cm}|p{2cm}|p{2cm}|p{2cm}|}
    \hline
    \textbf{Ítem} & \textbf{Descripción} & \textbf{Cantidad} & \textbf{Costo Unitario} & \textbf{Monto Total} \\
    \hline
    1 & Equipo de desarrollo de software & 1 & \$5000 & \$5000 \\
    \hline
    2 & Licencias de software especializado & 5 & \$1000 & \$5000 \\
    \hline
    3 & Equipamiento de oficina & 10 & \$500 & \$5000 \\
    \hline
    4 & Herramientas de diseño gráfico & 3 & \$1500 & \$4500 \\
    \hline
    \multicolumn{4}{|r|}{\textbf{Total}} & \textbf{\$19500} \\
    \hline
    \end{tabular}
    \caption{Costo de Máquinas, Herramientas y Equipos para el Desarrollo de una Página Web}
    \label{tab:costo_maquinas}
\end{table}

El costo asociado a las máquinas, herramientas y equipos es un factor crucial en la planificación de un proyecto como el desarrollo de una página web.

Esta tabla detalla los elementos esenciales necesarios para llevar a cabo dicho proyecto, desde el equipo de desarrollo de software hasta las licencias de software especializado y herramientas de diseño gráfico. La estimación de estos costos proporciona una visión detallada de los recursos necesarios para garantizar un entorno de trabajo óptimo y la implementación exitosa del proyecto. Así, esta información se convierte en un elemento clave en la gestión presupuestaria y en la toma de decisiones relacionadas con la inversión necesaria para llevar a cabo el desarrollo de la página web.

\section{Costo total de la implementación de la mejora}

El desglose de costos proporcionado refleja una evaluación detallada de los recursos necesarios para la implementación de la mejora en la presentación de servicios en la página web de la empresa Proefex. Estos costos consideran tanto los gastos asociados a la adquisición de equipos y herramientas especializadas, como los relacionados con la contratación de mano de obra cualificada y la obtención de materiales específicos para la ejecución del proyecto. El cálculo total representa la estimación financiera integral de la iniciativa, brindando una visión completa de la inversión requerida para llevar a cabo la mejora propuesta.

\begin{table}[!ht]
\centering
\begin{tabular}{|l|r|}
\hline
\textbf{Concepto} & \textbf{Costo Total} \\ \hline
Costo de máquinas, herramientas y equipos & \$19500 \\ \hline
Costo de mano de obra & \$25900 \\ \hline
Costo de materiales & \$1200 \\ \hline
\textbf{Total} & \textbf{\$46600} \\ \hline
\end{tabular}
\caption{Costo Total de Implementación de la Mejora}
\end{table}


\titlespacing{\chapter}{0pt}{120pt}{7pt}
\chapter{Evaluación Técnica y Económica de la Mejora}
\label{cap:evaluación}
\authoredby{B}

\section{Benefico técnico y/o econónico esperado de la mejora}

El beneficio técnico y económico esperado de la mejora en la presentación de servicios en la página web de Proefex se traduce en múltiples aspectos:

\textbf{Beneficio Técnico}

\begin{itemize}
\item \textbf{Mejora de Experiencia del Usuario:} La implementación de modelos 2D,
	animaciones interactivas y elementos tipo juego en la presentación de servicios
	en el sitio web mejorará la experiencia del usuario, haciéndola más atractiva y
	fácil de comprender.
\item \textbf{Aumento de Interacción:} La inclusión de elementos interactivos puede fomentar
la interacción del usuario con el contenido, lo que posiblemente aumente el tiempo
de permanencia en el sitio.
\item \textbf{Actualización Tecnológica:} La adopción de nuevas tecnologías en la presentación
de servicios coloca a la empresa en una posición más actualizada y competitiva en el
mercado.
\end{itemize}

\textbf{Beneficio Económico}

\begin{itemize}
\item \textbf{Atracción de Clientes Potenciales:} Una presentación más atractiva y dinámica
puede atraer a más clientes potenciales, lo que puede aumentar las conversiones y,
	  en última instancia, los ingresos.
\item \textbf{Retención de Clientes:} Una experiencia mejorada puede contribuir a la lealtad del
cliente y a una mayor retención, reduciendo posiblemente la tasa de rebote y aumentando
el retorno a largo plazo.
\item \textbf{Diferenciación Competitiva:} Al ofrecer una experiencia única y visualmente
impactante, la empresa puede destacar entre sus competidores, generando así un valor
adicional.
\end{itemize}

Estos beneficios, tanto en el ámbito técnico como económico, tienen el potencial de impactar positivamente en la percepción de la marca, el compromiso del usuario y, en última instancia, en el rendimiento financiero de la empresa.

\section{Relación Beneficio / Costo}

\cite{perez2023metodos} La relación Beneficio / Costo (B/C) se calcula mediante la fórmula:

\[
\text{B/C} = \frac{\text{Beneficio Total}}{\text{Costo Total}}
\]

\[
\text{B/C} = \frac{{19500 + 25900 + 1200}}{{19500 + 25900}} = \frac{46600}{45400} \approx 1.026
\]

Por tanto, la relación beneficio/costo calculada es aproximadamente 1.026, lo que indica que el proyecto tiene un rendimiento favorable en comparación con los costos incurridos. Esto sugiere una viabilidad positiva para la implementación de la mejora propuesta en términos de beneficios en relación con los costos asociados.


\titlespacing{\chapter}{0pt}{120pt}{7pt}
\chapter{Conclusiones}
\label{cap:concluciones}
\authoredby{B}

\section{Conclusiones respecto a los objetivos del Proyecto de Mejora}

El proyecto de mejora enfocado en la actualización de la presentación de servicios en la página web de Proefex ha logrado significativos avances. Se han implementado modelos 2D, animaciones interactivas y una experiencia de usuario similar a un juego, lo que ha permitido una visualización más dinámica y atractiva de los servicios ofrecidos.

A lo largo del desarrollo del proyecto, se ha observado un incremento en la interacción de los usuarios con la página, lo que sugiere un mayor interés y compromiso por parte de la audiencia. La adición de elementos interactivos ha generado una mayor retención de los visitantes, proporcionando una experiencia más inmersiva y efectiva.

Si bien los costos asociados con la implementación fueron considerables, el proyecto ha logrado generar un valor agregado notable para la empresa. La combinación de elementos visuales atractivos y una navegación más interactiva ha mejorado significativamente la presentación de los servicios, lo que probablemente resultará en un aumento de clientes potenciales y conversiones.

Es importante continuar monitoreando y analizando la recepción y el impacto de estas mejoras a largo plazo para evaluar su efectividad y justificar completamente la inversión realizada. Sin embargo, se espera que estas actualizaciones proporcionen una ventaja competitiva significativa para Proefex en su mercado objetivo.



\titlespacing{\chapter}{0pt}{120pt}{7pt}
\chapter{Recomendaciones}
\label{cap:recomendar}
\authoredby{B}

\section{Recomendaciones para la empresa respecto del Proyecto de Mejora}

Basado en los resultados obtenidos y la evolución del proyecto de mejora en la presentación de servicios en la página web, se sugieren las siguientes recomendaciones para Proefex:

1. Continuar el monitoreo: Mantener un seguimiento regular de las métricas clave de rendimiento web para evaluar la efectividad continua de las mejoras implementadas. Esto permitirá realizar ajustes y optimizaciones adicionales según sea necesario.

2. Recopilación de comentarios: Implementar sistemas de retroalimentación y encuestas para obtener comentarios directos de los usuarios sobre la nueva experiencia web. Esto proporcionará información valiosa para identificar áreas de mejora y satisfacer mejor las necesidades de los usuarios.

3. Optimización móvil: Asegurarse de que la experiencia del usuario en dispositivos móviles esté completamente optimizada, ya que un número significativo de usuarios acceden a la web a través de sus dispositivos móviles.

4. Capacitación del personal: Brindar capacitación continua al personal encargado de mantener y actualizar la página web. Esto garantizará que se utilicen todas las funciones y características implementadas de manera efectiva.

\newpage
5. Evolución constante: Estar al tanto de las últimas tendencias y tecnologías web para mantener la página actualizada y competitiva. La evolución constante es crucial para mantener el atractivo y la relevancia de la página web en un entorno digital cambiante.

Estas recomendaciones se centran en la optimización continua, el compromiso del usuario y la adaptación a las tendencias emergentes, lo que permitirá a Proefex mantener una presencia en línea sólida y atractiva.




% \cleardoublepage
% \phantomsection
% \addcontentsline{toc}{chapter}{\normalfont CONCLUSIONES}
% \titlespacing{\chapter}{0pt}{-30pt}{0pt}

\begin{conclusiones}
\label{cap:conclusiones}

Se desarrollan de acuerdo a los objetivos específicos y redactar en párrafos.

\end{conclusiones} %Inserta conclusiones

% \cleardoublepage
% \phantomsection
% \addcontentsline{toc}{chapter}{\normalfont RECOMENDACIONES}
% \titlespacing{\chapter}{0pt}{-30pt}{0pt}
\begin{recomendaciones}
\label{cap:recomendaciones}
 
Se presentan en relación con los objetivos específicos, dando cuenta de las consecuencias, orientaciones o medidas a realizarse.
    
\end{recomendaciones} %Inserta recomendaciones

%%%%%%%%%%%%%%%%%%%%%%%%%%%%%%%%%%%%%%%%%%%%%%%%%%%%%%%%%%%%%%%%%%%%%%
 \clearpage\phantomsection\addcontentsline{toc}{chapter}{\normalfont BIBLIOGRAFÍA}
 \renewcommand\bibname{BIBLIOGRAFÍA} %Inserta BIBLIOGRAFÍA EN APA7

\printbibliography 

% \bibliographystyle{apacite}
% \bibliography{mybib}


%%%%%%%%%%%
% \cleardoublepage
% \phantomsection
% \addcontentsline{toc}{chapter}{\normalfont ANEXOS}
% \begin{anexo}
\pagebreak
\hspace{0pt}
%\vfill
\newline
\newline
\newline
\newline
\newline
\begin{center}
\textbf{ANEXOS}
\end{center}
%\vfill
\hspace{0pt}
\pagebreak
\end{anexo} %Inserta Anexos
% \addtocontents{toc}{\protect\SpecialDocumentsTocHeading}

%%%%%%%%%%%%%
%CONFIGURACIÓN DE ANEXOS
%%%%%%%%%%%%%

% \appendix

\makeatletter
\def\@chapter[#1]#2{
  \ifnum \c@secnumdepth >\m@ne
  \refstepcounter{chapter}%
  \typeout{\@chapapp\space\thechapter.}%
  \addcontentsline{apx}{chapter}%NEW
  {\protect\numberline{\thechapter} \normalfont#1}%

  \else
  \addcontentsline{apx}{chapter}{#1}%NEW
  \fi
  \chaptermark{#1}%
  \addtocontents{lof}{\protect\addvspace{10\p@}}%
  \addtocontents{lot}{\protect\addvspace{10\p@}}%
  \if@twocolumn

  \@topnewpage[\@makechapterhead{#2}]%
  \else
  \@makechapterhead{#2}%
  \@afterheading
  \fi}
\makeatother



\renewcommand*{\thechapter}{\bfseries\arabic{chapter}.}

\titleformat{\chapter}{\bfseries}{\large Anexo \thechapter}{0.5em}{\normalfont #1}

\titlespacing*{\chapter}{0cm}{-\topskip}{0pt}[0pt]

%%%%%%%%%%%%%
%AQUÍ PUEDES AGREGAR TUS ANEXOS
%%%%%%%%%%%%%


\end{document}
